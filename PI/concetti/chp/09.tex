\chapter{Le interfacce utente}

\begin{itemize}
    \item Lo strumento \'e ci\`o che compie l'azione, l'interfaccia \'e ci\`o che serve per permette all'utente di guidare lo strumento nell'esecuzione dell'azione.
    \item \desc{User Interface} \'E lo spazio di un sistema dove avviene l'interazione tra uomo e macchina.
    \item \desc{Le interfacce sono organizzabili secondo livelli}
    \begin{itemize}
        \item \desc{Human Interface Device} La periferica grazie al quale l'utente interagisce con il sistema.
        \item \desc{Human Machine Interaction} Il concetto che astrae dall'HID. Si intende infatti tutto il sistema di interazione uomo-macchina.
        \item \desc{Human Computer Interaction} HMI con un computer.
    \end{itemize}
    \item \desc{Composite User Interface} Sono le interfacce che usano pi\`u di un senso.
    \item \desc{Graphical User Interface} Sono composte da interfacce grafiche e tattili.
    \item \desc{Multimeda User Interface} GUI + Sonoro.
    \item \desc{Catogorie di CUI}
    \begin{itemize}
        \item \desc{Standard} Utilizzano dispositivi standard come tastiere, mouse, monitor\dots
        \item \desc{Virtual} Creano un mondo virtuale che funge da interfaccia tra l'utente e la macchina.
        \item \desc{Augmented} Arricchiscono il mondo reale. L'interfaccia \'e un mix di contenuti reali e virtuali.
    \end{itemize}
    \item \desc{Qualia Interface} \'E un'interfaccia utente che interagisce con tutti i sensi.
\end{itemize}