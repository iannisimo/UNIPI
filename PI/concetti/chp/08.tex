\chapter{Errore umano}

\begin{itemize}
    \item \desc{Root Case Analysis} Consiste nell'indagare l'incidente finch\`e non si trova la singola causa che ne \'e l'origine.
    \item \desc{Errore unmano} Ogni deviazione dal comportamento appropriato. Gli errori si dividono in:
    \begin{itemize}
        \item \desc{Lapsus, slips} Quando si intende eseguire un'azione e si finisce per farne un'altra. I lapsus si hanno nelle fasi viscerali e comportamentali dell'azione. I lapsus si dividono in:
        \begin{itemize}
            \item \desc{Lapsus di azione} Si esegue un'azione sbagliata.
            \item \desc{Lapsus di memoria} Si dimentica di eseguire l'azione o di valutarne i risultati.
        \end{itemize}
        \item \desc{Errori cognitivi, mistakes} Si ha un errore cognitivo quando \'e sbagliato il goal o lo scopo. Gli errori si hanno nelle fasi riflessive dell'azione. Gli errori cognitivi si suddividono in:
        \begin{itemize}
            \item \desc{Regola sbagliata} Lo scopo \'e giusto ma viene scelto un corso d'azione sbagliato.
            \item \desc{Conoscenza sbagliata} La diagnosi della situazione \'e sbagliata.
            \item \desc{Dimenticanza} Si hanno quando ci si dimentica qualche passaggio al momento di fissare gli obbiettivi.
        \end{itemize}
    \end{itemize}
    \item \desc{Prevenzione dell'errore}
    \begin{itemize}
        \item Comprendere le cause dell'errore
        \item Effettuare controlli di sensibilit\'a
        \item Rendere possibile annullare le azioni o rendere pi\`u difficile ci\`o che non pu\`o essere annullato
        \item Rendere pi\`u semplice la scoperta e comprensione degli errori
        \item Aiutare l'utente a compiere correttamente l'azione
    \end{itemize}
    \item Gli errori cognitivi dipendono da informazioni ambigue o poco chiare sullo stato attuale del sistema e dalla mancanza di un buon modello concettuale.
    \item Le interruzioni sono lam causa principale degli errori, soprattutto dei lapsus.
    \item Riducendo i passaggi dell'azione \'e possibile diminuire il costo di attenzione necessario per riprendere la concentrazione dopo essere stati interrotti.
    \item I feedback errati spesso vengono silenziati o ignorati, facendo perdere di significato anche quelli utili per il raggiungimento dello scopo.
    \item Per prevenire errori \'e possibile utilizzare:
    \begin{itemize}
        \item \desc{Constraints} Segregare i controlli e separare i moduli.
        \item \desc{Undo} Annullare le operazioni sbagliate.
        \item \desc{Messaggi di errore e di conferma} Inutili, meglio mostrare l'azione da eseguire insieme all'oggetto interessato.
        \item \desc{Controlli di sensibilit\'a} Controllare che l'operazione sia sensibile o ragionevole.
    \end{itemize}
    
\end{itemize}