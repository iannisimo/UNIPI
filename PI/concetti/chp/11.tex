\chapter{Metodi e Strumenti Per l'Innovazione}

\begin{itemize}
    \item \desc{Innovazione Incrementale} Punta al mantenimento della competivit\'a aziendale.
    \item \desc{Disruptive Innovation} \'E un prodotto o servizio che cambi radicalmente il modo in cui si fanno le cose. Si punta quindi a conquistare quelle nicchie di clientela che risultano ancora irraggiungibili tramite prodotti esistenti. Non si pu\`o fare innovazione dirompente senza una forma di pensiero e progettazione antropocentrica.
    \item \desc{Human Centered Design Process} Considera l'attivit\'a umana come un ombrello al di sotto del quale interagiscono i fattori umani, tecnologici, e sociali. Poggia su un percorso di progettazione diviso in tre fasi:
    \begin{itemize}
        \item \desc{Ispirazione} Come migliorare uno strumento osservando il modo in cui una persona lo utilizza.
        \item \desc{Ideazione} Ragionare su pi\`u idee possibili rimanendo focalizzati sui bisogni e necessit\'a dei destinatari. Il prototipo serve come punto di partenza per un confronto con i destinatari.
        \item \desc{Implementazione} Dopo i passi precedenti \'e possibile arrivare ad implementare il prodotto.
    \end{itemize} 
    \item \desc{Design Thinking} \'E centrato sulle persone e si basa sull'abilit\'a di integrare capacit\'a analitiche con attitudini creative. Ha come obbiettivo quello di trovare una soluzione innovativa ad un problema tenendo conto del gradimento. L'approccio Design Thinking pone nel vertice in alto le persone.
    \item Il processo di sviluppo mediante Design Thinking pu\`o essere suddiviso in cinque fasi: Empathize, Define, Ideate, Prototype, Test.
    \item Quindi lo HCD \'e un mindset mentre il DT \'e un metodo di lavoro che consente di sviluppare prodotti centrati sull'utente.
\end{itemize}