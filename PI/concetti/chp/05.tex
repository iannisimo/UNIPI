\chapter{Principi fondamentali dell'interazione}

\begin{itemize}
    \item \'E grazie all'esperienza che si crea la tonalit\'a del ricordo che conserviamo e associamo agli oggetti con cui abbiamo interagito.
    \item Quando la tecnologia si comporta in maniera inaspettata gli utenti provano emozioni negative.
    \item Cognizione ed emozione sono profondamente legate
    \item La visibilit\'a di ottiene tramite:
    \begin{itemize}
        \item \desc{Affordance} \'E una relazione tra un oggetto e un utente; Le affordance devono essere percepibili.
        \item \desc{Signifier} \'E un modo per indicare dove effettuare un'azione; I signifier possono essere voluti o accidentali.
        \item \desc{Mapping} Permette di relazionare i signifier alle affordance disponibili; Il mapping naturale \'e il migliore perch\`e non va contro le relazioni gi\'a presenti nel cervello umano.
        \item \desc{Feedback} \'E la comunicazione del risultato di un'azione; Deve essere immediato, informativo ed essenziale.
    \end{itemize}
    \item \desc{Modello concettuale} \'E una descrizione sintetica delle funzionalit\'a di un sistema; Esprime come il designer vuole che l'utente percepisca il prodotto.
    \item \desc{Modello mentale} \'E un modello concettuale nella mente dell'utente che rappresenta il modo in cui, secondo lui, funzionano le cose.
    \item Pi\`u \'e grande la differenza tra il modello mentale e quello concettuale, pi\`u l'utente far\'a fatica ad utilizzare il sistema.
    \item \desc{Immagine di sistema} \'E tutto ci\`o che si pu\`o inferire sul prodotto dati documentazione, istruzioni, significanti \dots
\end{itemize}