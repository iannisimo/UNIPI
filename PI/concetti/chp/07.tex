\chapter{How do people do things}

\begin{itemize}
    \item \desc{Golfo dell'esecuzione} Corrisponde allo sforzo necessario per capire come raggiungere uno scopo; Per superarlo si usano significanti, constraints e mapping.
    \item \desc{Golfo della valutazione} Corrisponde allo sforzo necessario per interpretare lo stato fisico del dispositivo e capire fino a che punto sono state realizzate le intenzioni iniziali; Per superarlo si usano i feedback.
    \item \desc{I sette stadi dell'azione}
    \begin{itemize}
        \item Goal
        \item Plan
        \item Specify
        \item Perform
        \item Perceive
        \item Interpret
        \item Compare
    \end{itemize}
    \item Gli stadi dell'azione possono essere associati a tre livelli di processing mentale
    \begin{itemize}
        \item \desc{Viscerale} Subconscio
        \item \desc{Comportamentale} Subconscio
        \item \desc{Riflessivo} Conscio
    \end{itemize}
    \item \desc{Feedforward} Sono tutte le informazioni necessarie per le fasi attuative.
    \item \desc{Feedback} Comprende le informazioni necessarie per le fasi percet\-tive; \'E dato dall'immediato cambiamento di stato del sistema.
    \item \desc{Sette principi fondamendali del design}
    \begin{itemize}
        \item Visibilit\'a
        \item Feedback
        \item Modello concettuale
        \item Affordance
        \item Significanti
        \item Mapping
        \item Vincoli
    \end{itemize}
\end{itemize}