\chapter{Human Centered Design}

\begin{itemize}
    \item Nel design antropocentrico si mette l'utente al centro del processo.
    \item \desc{Human Centered Design} \'E una metodologia di progettazione che parte dai bisogni umani, adattandola da essi. \'E un approccio di design orientato allo sviluppo di sistemi interattivi focalizzati sull'utente.
    \item Bisogna focalizzare l'attenzione su ci\`o che potrebbe andare storto, cos\`i da ridurre la frustrazione quindi la negativit\'a verso il prodotto.
    \item L'obbiettivo dello \m{HCD} deve essere quello di creare nell'utente empatia verso il sistema.
    \item Processo di \m{HCD}
    \begin{itemize}
        \item Specificare il contesto d'uso
        \item Specificare i requirements
        \item Progettare la Soluzione
        \item Testare e valutare
    \end{itemize}
    \item Nello sviluppo software diventa indispensabile abilitare sistemi di tracciamento dell'utente finalizzati alla produzione di statistiche di utilizzo
    \item \desc{L'usabilit\'a} \'E la disciplina che regola la costruzione del sistema in base alle esigenze dell'utente, cercando di semplificare la sua esperienza di navigazione.
\end{itemize}