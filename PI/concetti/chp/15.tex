\chapter{GUI Design}

\begin{itemize}
    \item Si basa sull'anticipare i bisogni dell'utente.
    \item \desc{Strutture delle interfacce}
    \begin{itemize}
        \item \desc{Gerarchiche} O a albero; Parti dal generale (radice), e arrivi tramite le categorie (i nodi), alle informazioni cercate (le foglie)
        \item \desc{Sequenziale} Step by step; Puoi muoverti avanti e indietro nella catena
        \item \desc{Matrix} Ogni utente pu\`o scegliere il proprio percorso; open world, task secondari
        \item \desc{Database} I collegamenti sono dettati dalla struttura del database.
    \end{itemize}
    \item Mantenersi aperti al progresso (possibilit\'a di modificare / aggiungere sezioni)
    \item Non richiedere troppi passi per arrivare \textit{alle foglie desiderate}
    \item \desc{Architettura dell'informazione} Organizzare, strutturare ed etichettare i contenuti in modo efficiente ed eco-sostenibile.
    \item Si basa sull'interdipendenza tra: contesto, contenuti, e utenti.
    \item \desc{Schemi organizzativi}
    \begin{itemize}
        \item \desc{Esatti} Informazioni divise per regole oggettive; ordine alfabetico, cronologico, geografico \dots
        \item \desc{Soggettivi} Informazioni divise in base all'utente; Netflix: Audience; File explorer: metaforico \dots
    \end{itemize}
    \item \desc{Document Object Model} \'E un'interfaccia che tratta documenti (html, xml \dots) come un albero, in cui ogni nodo \'e parte del documento.
    \item I browser creano una rappresentazione ad oggetti di un documento HTML per permettere a JS di accedervi e manipolarlo
\end{itemize}