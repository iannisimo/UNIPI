\chapter{UX Design}

\begin{itemize}
    \item \desc{Personas} \'E l'archetipo di uno dei possibili utenti. Le tecniche per l'identificazione delle \textit{personas} sono:
    \begin{itemize}
        \item Task Analysis
        \item Feedback
        \item Prototipazione
    \end{itemize}
    \item \desc{Principio di Pareto} Concentrarsi sul 20\% degli utenti che utilizzer\`a l'80\% del prodotto.
    \item \desc{Requirements} \'E un servizio o una caratteristica che soddisfa un bisogno di un utente. Utilizzando le personas \'e molto pi\`u semplice individuare i requirements.
    \item \desc{Requirements Driven Development} \'e un approccio complesso e oneroso e va in contrasto con il metodo agile. \'E bene seguirlo una volta definiti i requirements.
    \item \desc{Requirements Funzionali} Descrivono quali funzionalit\'a \textbf{deve} avere il software.
    \item \desc{Requirements Non Funzionali} Specificano o tratti qualitativi del prodotto.
    \item \desc{User Story} \'E una breve descrizione che identifica l'utente insieme al suo obbiettivo e le sue necessit\'a. Determina chi \'e, di cosa ha bisogno, e perch\`e ne ha bisogno. Tipicamente ci sono pi\`u user stories per ogni personas, ma non il contrario. Una user story \'e un requirement espresso dalla prospettiva del cliente.
    \item \desc{Scenarios} Estendono la user story andando a descrivere anche quali motivazioni hanno portato l'utente ad usare il software e come egli si comporter\'a nel suo utilizzo. Per definire gli scenarios \'e necessario mapparli avendo gia definito le personas e le relative user stories, e individuando per ogni personas un key task.
    \item Ci sono tre metodi principali per scrivere gli scenarios:
    \begin{itemize}
        \item Piccoli goal o task-oriented scenarios
        \item Elaborated scenarios
        \item Full scale task scenarios
    \end{itemize}
    \item \desc{Use Cases} Consistono della completa narrativa di quali azioni l'utente compie per svolgere lo scenario. Uno scenario si concentra su uno o pi\`u attori, mentre un caso d'uso \'e incentrato su una persona. Con casi d'uso ben fatti si pu\`o passare direttamente all'implementazione.
\end{itemize}