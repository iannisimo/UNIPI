\chapter{Tecniche di Test e Prototipazione delle Interfacce}

\begin{itemize}
    \item \desc{Lost in translation} L'idea \'e un'astrazione soggettiva. Nel momento in cui si prova a comunicarla si incappa in questo problema.
    \item \desc{Problema della predizione} \'E molto difficile per una persona prevedere se un'idea potrebbe essere di gradimento ad un utente.
    \item Bisogna uscire dal Thoughtland e muoversi verso l'Actionland.
    \item I prodotti del Thoughtland sono idee, domande, e opinioni.
    \item I prodotti dell'Actionland sono artefatti, azioni, e dati.
    \item \desc{Pretotipo} \'E un semplice mockup del prodotto che si vorrebbe sviluppare. Un pretotipo costa molto meno sia in termini di tempo che di denaro.
    \item \desc{Tipo di Pretotyping}
    \begin{itemize}
        \item \desc{Fake Door} Pubblicizzare un prodotto che ancora non esiste per tracciare il grado di approvazione degli utenti.
        \item \desc{Mechanical Turk} Front-end pronto, Back-end powered by man-power.
        \item \desc{Impersonator} Skinnare un prodotto gia esistente.
        \item \desc{Pinocchio} Un prototipo non funzionale ma con dimensioni e forme reali.
        \item \desc{One Night Stand} Versione funzionante ma non scalabile
        \item \desc{Facade} Simile ad un Impersonator ma fa apparire stabilit\'a nell'azienda.
    \end{itemize}
    \item \desc{Minimum Viable Product} \'E versione minimale del prodotto che implementa solo i requirements funzionali.
\end{itemize}