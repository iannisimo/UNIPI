\chapter{Vincoli}

\begin{itemize}
    \item I vincoli limitano l'insieme delle azioni possibili. Si possono avere:
    \begin{itemize}
        \item \desc{Fisici} Si affidano a propriet\'a del mondo fisico
        \item \desc{Culturali} Si affidano alle abitudini culturali.
        \item \desc{Semantici} Si affidano al significato della situazione.
        \item \desc{Logici} Dettati dalla semplice e pura logica umana.
    \end{itemize}
    \item Mapping forti possono diventare vincoli logici.
    \item \desc{Funzioni obbliganti} Sono una forma di vincolo fisico.
    \begin{itemize}
        \item \desc{Interlock} Obbliga ad eseguire una serie di operazioni nella sequenza dovuta prima di avviare la sequenza richiesta.
        \item \desc{Lock-in} Mantiene attiva una funzione impedendo che venga interrotta prematuramente.
        \item \desc{Lock-out} Impedisce l'ingresso in uno spazio pericoloso o impedisce che succeda qualcosa.
    \end{itemize}
    \item \desc{Activity-Centered Control} Le funzioni degli elementi interattivi cambiano a seconda dello \textit{stato}
\end{itemize}