\chapter{Progettare l'interazione tra uomo e macchina}

\begin{itemize}
    \item \desc{Design} Sia il processo di progettazione e pianificazione che l'output stesso di questo processo.
    \item \desc{Computational Thinking} Pensare a diversi livelli di astrazione; \'E un processo mentale che consente di risolvere problemi di varia natura seguendo metodi e utilizzando strumenti specifici.
    \item Il design dell'interazione e il pensiero computazionale non sono mutualmente esclusivi.
    \item \desc{Interaction Design} \'E l'attivit\'a di progettazione volta a rendere macchine e servizi utilizzabili dagli utenti e non soltanto dai propri creatori; I bisogni degli utenti devono essere al centro del processo di sviluppo.
    \item \desc{HCI} Human-Computer Interaction.
    \item \desc{HMI} Human-Machine Interaction.
    \item \desc{OBBBiettivi di HCI e HMI} Facilitare l'uso dei sistemi del mondo \m{IT}.
    \item \desc{Design di Prodotto} Processo di progettazione di beni e servizi con lo scopo di essere utilizzati da pi\`u utenti possibile; Il product designer, in maniera molto grossolana, deve stilare una lista in cui descrive: \{Problema da assolvere; Funzionalit\'a principale; Soluzioni esistenti; Soluzione proposta\}.
    \item \desc{User Experience Design} Processo volto ad aumentare la soddisfazione del cliente migliorando l'usabilit\'a del Prodotto.
    \item \desc{UX Designer} Ha l'obbiettivo di migliorare l'esperienza dell'utente e di ridurre al minimo le sensazioni di frustrazione e delusione.
    \item \desc{User Interaction Design} Studia come le persone interagiscono con la tecnologia.
    \item \desc{UI Designer} Progetta l'aspetto estetico e la struttura dell'inter\-faccia
    \item \desc{User-Interaction Designer} Progetta l'aspetto estetico e la struttura dell'interfaccia; produce un wireframe e una serie di linee guida che verranno poi seguite dagli sviluppatori. L'interfaccia viene implementata solo alla fine del percorso di progettazione.
\end{itemize}