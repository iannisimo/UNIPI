\chapter{Lo Strato di Rete: IP}
\section{Concetti generali}
Lo strato di rete si occupa dello scambio dei datagrammi tra gli hosts.
Offre servizi allo strato superiore \textit{(trasporto)} e utilizza i servizi di quello inferiore \textit{(collegamento)}.
Offre un'astrazione che consente a host e reti diverse di funzionare, dal punto di vista logico, come una singola rete.
\subsection{Passaggi della comunicazione}
\paragraph{Mittente}
L'entità a livello di rete riceve i segmenti e li incapsula in datagrammi prima di essere spediti verso il prossimo nodo della connessione (host o router).
\paragraph{Nodo intermedio}
Il router esamina i campi intestazione dei datagrammi che riceve da un collegamento in ingresso, e li inoltra al collegamento in uscita.
\paragraph{Destinatario}
I datagrammi arrivano allo strato di rete del destinatario, che li consegna al rispettivo strato di trasporto (demux TCP o UDP).
\section{Internet Protocol (IP)}
Descritto dalla RFC-791, e' un protocollo di tipo connection-less (Send-and-Pray, QoS non garantito, \dots).
\subsection{Protocol Data Unit (PDU)}
\begin{center}
    \begin{tabular}{|m{12mm}|m{12mm}|m{12mm}|c|}
        \hline
        Frame\newline header&IP\newline header&TCP\newline header&\hspace{10mm}User data\hspace{10mm}\space\\
        \hline
    \end{tabular}
\end{center}
\subsection{Servizi offerti}
\paragraph{Inoltro (forwarding)}
Permette di instradare verso il corretto collegamento di uscita.
Per fare cio' il router legge l'indirizzo di destinazione dal campo di intestazione e lo confronta con una \bluetext{tabella di inoltro}, la quale indica la giusta interfaccia di uscita per quell'indirizzo.
\paragraph{Instradamento (routing)}
E' il processo decisionale che sceglie il percorso migliore tra mittente e destinazione.
\paragraph{Indirizzamento}
Strumento per identificare gli host nell'internet.
\paragraph{Modello datagram} Permette la segmentazione / riunificazione dei pacchetti, il controllo degli errori (nell'header), e la verifica TTL.
\subsection{Formato dei datagrammi}
\begin{center}
    \begin{tabular}{|c|c|}
        \hline
        Intestazione&\hspace{15mm}Dati\hspace{15mm}\space\\
        \hline
    \end{tabular}
\end{center}
\paragraph{Intestazione} e' lunga 20-60Byte ed e' formata da:
% TODO: Se mai mi verra' voglia aggiungerò le dimensioni dei pezzetti di pipi qui sotto.
% 4 4 8 16 16 3 13 8 8 16 32 32 0-40
\begin{description}
    \item[ver] Versione del protocollo IP (IPv4, IPv6).
    \item[head. len] Lunghezza dell'header in parole di 4Byte.
    \item[type of service] 
    \item[lenght] Lunghezza del datagramma in Byte.
    \item[16-bit identifier] Per frammentazione.
    \item[flags] Per frammentazione.
    \item[fragment offset] Per frammentazione.
    \item[time to live] Massimo numero di \textit{hop} rimanenti (decrementato ad ogni router).
    \item[upper layer] Protocollo di trasporto.
    \item[header checksum] Calcolato ad ogni router.
    \item[32 bit source IP address] Indirizzo sorgente.
    \item[32 bit destination IP address] Indirizzo destinazione.
    \item[options] Facoltativo.
    \item[data] Lunghezza variabile, solitamente segmento TCP o UDP.
\end{description}
\subsection{Frammentazione}
Diversi protocolli di trasferimento possono piu' o meno dati in un solo frame. Il massimo e' determinato dal Maximum Transfer Unit (MTU).
Cio' significa che un router potrebbe ricevere un frame piu' grande dell'MTU della rete verso la quale deve inoltrarlo; in questi casi il router deve dividere il datagramma in piu' datagrammi piu' piccoli detti frammenti.
Il riassemblaggio dei frammenti viene effettuato dall'entità rete del destinatario.
Se uno o piu' frammenti vengono persi l'intero messaggio viene scartato.

Per riordinare i frammenti vengono usati 3 campi intestazione:
\begin{description}
    \item[Identificatore] e' un valore associato alla coppia <mittente, destinatario> che identifica il datagramma.
    \item[Offset] indica la posizione relativa come multiplo di 8Byte (Il primo frammento ha offset 0).
    \item[Flag] Serve ad identificare l'ultimo frammento:
    \begin{description}
        \item[bit 0] \textit{reserved}, sempre a 0 per il momento.
        \item[bit 1] \textit{do not fragment}, vale 1 se il pacchetto non puo' essere frammentato.
        \item[bit 2] \textit{more fragments}, vale 1 se il pacchetto non e' l'ultimo del frammento.
    \end{description} 
\end{description}
La frammentazione e' un processo critico perché richiede risorse ai nodi intermedi e destinazione e introduce ritardi perciò e' preferibile non usarla (impostare il \textit{do not fragment} a 1 in IPv4, IPv6 non supporta la frammentazione).
\section{Indirizzamento IPv4}
Gli indirizzi IPv4 sono costituiti da 4Byte.
Ogni host ha un indirizzo univoco diviso in 2 parti
\begin{tabular}{|c|c|}
    \hline
    \hspace{15mm}Prefisso\hspace{15mm}\space&\hspace{5mm}Suffisso\hspace{5mm}\space\\
    \hline
\end{tabular}\\
Il Prefisso individua la rete metre il suffisso individua il singolo host.
\subsection{Classful addressing}
Per riconoscere un indirizzo c'e' bisogno di indicare la lunghezza del prefisso (e di conseguenza del suffisso).

Un sistema e' il \bluetext{classful addressing} che definisce delle classi e le rispettive divisioni P/S.
Per riconoscere le classi si imposta il primo byte dell'indirizzo in un range di valori prefissato.

\begin{tabular}{||c c c||}
    \hline
    Classe & Primo Byte & Pref. Len.\\
    \hline\hline
    A & 0-127 & 8bit\\
    B & 128-191 & 16bit\\
    C & 192-223 & 24bit\\
    D & 224-239 & Non applicabile\\
    E & 240-255 & Non applicabile\\
    \hline
\end{tabular}
\subsection{Classless addressing}
Utilizza la notazione Classless Interdomain Routing (CIDR) cioe':
\begin{quote}
    byte.byte.byte.byte/n
\end{quote}
con n pari alla lunghezza del prefisso.

Dato un indirizzo \bluetext{a.b.c.d/n} si definisce la \bluetext{Subnet mask} come valore da 32bit con i primi \bluetext{n} bit a sinistra impostati a 1.
Effettuando una Bitwise-AND tra un indirizzo e la maschera si ottiene l'indirizzo della rete.

\section{Assegnazione blocchi di indirizzi}
Gli indirizzi IP \textit{"globali"} sono gestiti da ICANN, che li assigna agli ISP in blocchi. Sono poi gli ISP ad assegnare sottoblocchi di essa ai suoi clienti.
\subsection{Aggregazione di indirizzi}
Per rendere piu' efficiente l'instradamento, gli ISP possono assegnare vari blocchi di indirizzi a piu' clienti per poi aggregarli in un singolo blocco da annunciare alla rete.
\subsection{Indirizzi speciali}
\begin{description}
    \item[0.0.0.0] This host, usato come indirizzo sorgente quando un host non sa il proprio indirizzo.
    \item[255.255.255.255] Limited-broadcast, usato per inviare un datagramma a tutti i dispositivi all'interno della rete locale.
    \item[127.0.0.1] Loopback, il datagramma non lascia l'host locale.
    \item[10.0.0.0/8] 
    \item[127.16.0.0/12] 
    \item[192.168.0.0/16]
    \item[169.254.0.0/16] Indirizzi privati, riservati per le reti locali.
    \item[224.0.0.0/4] Indirizzi multicast.
\end{description}
\subsection{DHCP}
Un indirizzo IP puo' essere attribuito a un host tramite configurazione manuale (l'host o chi lo gestisce configura l'indirizzo e le altre informazioni di servizio), oppure tramite DHCP (l'host ottiene tutte le informazioni automaticamente).

Il Dynamic Host Configuration Protocol e' un protocollo client-server che assegna indirizzi IP in modo dinamico.
Appena un host si connette alla sottorete da inizio alla procedura di assegnazione comunicando con il server DHCP della sottorete su cui e' inserito.
\section{Network Address Translation}
L'accesso di una rete privata su Internet avviene tramite un router abilitato alla NAT.
Tutto il traffico in uscita dal router di accesso ha come indirizzo IP quello pubblico del router.
Tutto il traffico in ingresso alla subnet e' indirizzato verso l'IP pubblico del router.

Il router mantiene una tabella di traduzione NAT, le cui righe contengono associazioni del tipo:
\begin{quote}
    (IP privato, porta) $\leftrightarrow$ (IP pubblico, porta)
\end{quote}
\section{Internet Control Message Protocol}
L'ICMP e' usato da host e router per scambiarsi messaggi di errore o altre situazioni che richiedono intervento.
Sono incapsulati all'interno di datagrammi IP ma il protocollo viene comunque considerato parte dello stato di rete.
I pacchetti ICMP vengono instradati dai router prima dei pacchetti IP ordinari.
Sono relativi solo al frammento 0 in caso di frammentazione del pacchetto.
\subsection{Tipi di messaggio}
I messaggi ICMP si dividono in
\begin{itemize}
    \item Messaggi di segnalazione errore.
    \item Messaggi di richiesta/risposta.
\end{itemize}
L'header ICMP e' formato da un campo Tipo (8bit) e un campo Codice (8bit).
\begin{center}
    \begin{tabular}{||c c r||}
        \hline
        Tipo & Codice & Descrizione\\
        \hline\hline
        0 & 0 & Echo replay. (ping)\\
        3 & 0 & Dest. network unreachable.\\
        3 & 1 & Dest. host unreachable.\\
        3 & 2 & Dest. protocol unreachable.\\
        3 & 3 & Dest. port unreachable.\\
        3 & 6 & Dest. network unknown.\\
        3 & 7 & Dest. host unknown.\\
        4 & 0 & Source quench. (congestion control)\\
        8 & 0 & Echo request.\\
        9 & 0 & Router advertisement.\\
        10 & 0 & Router discovery.\\
        11 & 0 & TTL expired.\\
        12 & 0 & Bad IP header\\
        \hline
    \end{tabular}
\end{center}
\subsection{Ping} 
PING e' un'applicazione di ICMP utilizzata per verificare lo stato di funzionamento di un altro host.
Viene inviata una richiesta eco (08,00) e si attende la risposta dell'host destinazione (00,00).
Essa fornisce anche misure dell'RTT e permette in maniera molto grossolana di misurare l'affidabilità e la congestione dei router tra due host inviando una sequenza di messaggi richiesta-risposta.
\subsection{Traceroute}
Traceroute si basa sui messaggi di errore (tempo scaduto e porta non raggiungibile).
Invia datagrammi UDP con valori di TTL crescenti (partendo da zero); quando il TTL raggiunge 0, il router invia un messaggio di errore.
L'host conta il tempo di ritorno dell'errore per trovare il RTT di ciascun router del percorso.
\section{IPv6}
\begin{center}
    \begin{tabular}{||l | r||}
        \hline
        Motivazione & Soluzione\\
        \hline
        \hline
        Esaurimento indirizzi 32bit & Indirizzi a 128bit\\
        \hline
        Velocizzare elaborazione e forwarding & Lunghezza fissa header\\
        \space & Eliminato checksum\\
        \hline
        Risparmiare costo frammentazione ai router & ICMPv6 msg 'packet too big'\\
        \hline
        Facilitare QoS & Introdotta 'flow label'\\
        \hline
    \end{tabular}
\end{center}
\subsection{Dual stack}
E' possibile mascherare un datagramma IPv6 (header + payload) come payload di un datagramma IPv4.
\begin{center}
    \begin{tabular}{c c c}
        \space & \space & header IPv4\\
        header IPv6 & $\rightarrow$ & \small header IPv6\\
        payload & \space & \small payload\\
    \end{tabular}
\end{center}