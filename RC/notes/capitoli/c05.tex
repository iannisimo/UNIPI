\chapter{Lo Strato di Trasporto: Introduzione}
\section{Obbiettivo}
Realizza una comunicazione logica tra processi su terminali diversi.
Cio' significa che essi comunicano come se ci fosse un collegamento diretto tra i due.
Lo strato applicazione trasmette dati allo strato di trasporto, il quale poi comunica con gli strati inferiori per trasmetterli all'host di destinazione.
\section{Servizi offerti}
Lo strato di trasporto si occupa principalmente di multiplexing/demultiplexing e del controllo degli errori.
\subsection{Protocolli}
In base al protocollo utilizzato per la connessione, lo strato di trasporto si occupa anche di altri fattori.
I due protocolli {\tiny(principali?)} sono:
\begin{description}
    \item[TCP] (RFC 793) Si occupa della gestione della connessione, dei controlli di flusso e congestione, e assicura una consegna affidabile.
    \item[UDP] (RFC 768) Protocollo connection-less, non affidabile ma piu' veloce. 
\end{description}
\subsection{Multiplexing/Demultiplexing}
\paragraph{Multiplexing} Lo strato di trasporto si occupa di \textit{accorpare} i flussi di dati e spedirli verso la loro destinazione insieme ad una \textit{busta} di trasporto.
\paragraph{Demultiplexing} Lo strato di trasporto si occupa di ricevere i dati dalla rete e instradarli verso i processi desiderati.
\paragraph{} Entrambi si basano sul socket addres (IP + porta) per identificare i processi.
In base al protocollo usato, lo strato di trasporto si occupa anche del 
\paragraph{Demultiplexing connectionless - UDP}
La \textit{busta} contiene solamente la socket address del destinatario. I datagrammi provenienti da host differenti vengono consegnati tutti alla stessa socket di destinazione.
\paragraph{Demultiplexing orientato alla connessione - TCP}
La \textit{busta} contiene le socket address di mittente e destinatario. Un host puo' supportare piu' socket contemporaneamente (p.e. server Web). Lo strato di trasporto puo' quindi inviare i dati alla socket appropriata in base al mittente.
\subsection{Le porte}
Ogni processo che intende comunicare con la rete (punto di demultiplexing) e' identificato tramite un socket address.
La {\color{blue}porta} e' un valore u-int16(0-65535). 
I range standard sono:
\begin{description}
    \item[System ports] (0-1023) Assegnate da IANA, identificano i processi server.
    \item[User ports] (1024-49151) Assegnate da IANA, non e' possibile la duplicazione.
    \item[Dynamic ports] (49152-65535), Non assegnate da IANA, muoiono al momento della chiusura della connessione.
\end{description}
\subsection{Well known ports}
\begin{tabular}{c c}
        20/tcp & ftp-data \\
        21/tcp & ftp \\
        22/tcp & ssh \\
        23/tcp & telnet \\
        25/tcp & smtp \\
        53/udp & dns \\
        53/tcp & dns \\
        69/udp & tftp \\
        80/tcp & www-http \\
        110/tcp & pop3 \\
        119/tcp & nntp \\
        161/udp & sntp \\
        220/tcp & imap3 \\
        510/udp & RIP \\
        3306/tcp & mysql
\end{tabular}