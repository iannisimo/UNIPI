\chapter{Lo Strato Applicativo: URI-HTTP}
\section{Applicazioni di rete}
Sono applicazioni formate da processi distibuiti comunicanti; Cio' significa che i processi possono essere eseguiti su host diversi sulla rete.
Sullo stesso host piu' processi possono comunicare anche attraverso la {\color{blue}comunicazione inter-processo} definita dal sistema operativo (quindi esterna alla rete).
Due processi su macchine diverse comunicano tramite la rete con dei \textbf{\color{blue} messaggi}.
Grazie alla statificazione dei livelli, i livelli applicazione comunicano tra loro come se esistesse un collegamento diretto tra i due.
\section{Protocollo dello Strato di Applicazione}
Esso definisce:
\begin{itemize}
    \item i tipi di messaggi scambiati
    \item la sintassi dei messaggi \textit{(i campi)}
    \item la semantica dei campi \textit{(il significato)}
    \item le regole per l'invio dei messaggi \textit{(quando e come)}
\end{itemize}
\newpage
\section{Paradigmi}
\begin{description}
    \item[Client-Server] I server \textit{(pochi ma potenti)} offrono un servizio e sono sempre in attesa di richieste.
    \item[Peer-tp-Peer (p2p)]  I peer offrono e richiedono servizi contemporaneamente.
    \item[Misto] Programmi che utilizzano entrambi i paradigmi per utilizzi diversi \textit{(Skype, \dots)} 
\end{description}
\section{Application Programming Interface (API)}
E' un insieme di regole da rispettare per utilizzare le risorse di un servizio.
Per esempio, se un processo vuole inviare un messaggio sulla rete, utilizza delle chiamate al sistema operativo che comunica con gli strati inferiori dello stack TCP-IP attraverso l'{\color{blue}Interfaccia Socket}.
\paragraph{L'Interfaccia Socket} E' un'API che collega gli strati applicazione e trasporto.
Grazie ad essa un programmatore che intende sviluppare un'applicazione con accesso a Internet non deve preoccuparsi di tutte le \textit{formalita'} per la connessione.
Per identificare un processo sulla rete si utilizza una coppia:
\begin{center}
    \textbf{\color{blue} $<$Indirizzo IP (32b) + numero di porta (16b)$>$}
\end{center}
\section{Uso dei servizi di trasporto}
Nel livello di trasporto sono previsti dule protocolli principali:
\begin{description}
    \item[TCP] connection-oriented: client e server devono \textit{salutarsi} prima di iniziare a comunicare; c'e' controllo degli errori sui messaggi.
    \item[UDP] connection-less: \textbf{\huge\color{red}CHAOS!} Non c'e' nessun controllo, il mittente tira i messaggi sulla rete e suppone che il destinatario lo riceva.
\end{description}
UDP e' utile se si ha bisogno di un alto troughput e/o un basso ritardo, ma solo se non ci interessa un trasferimento affidabile al 100\% dei dati \textit{(streaming, videogiochi, \dots)}.
\newpage
\section{Web e HTTP}
\subsection{Il Web}
Una pagina Web consiste di:
\begin{itemize}
    \item Un file HTML
    \item Diversi oggetti referenziati indirizzabili tramite una URL (Uniform Reference Locator)
\end{itemize}
\subsection{Uniform Resource Identifier}
E' una forma generale per identificare una risorsa sulla rete.
Ci sono due tipi di URI:
\begin{description}
    \item[Uniform Resource Locator (URL)] Risorse identificate tramite il meccanismo di accesso \textit{(HTTP, FTP, \dots)}
    \item[Uniform Resource Name (URN)] Identificatore globalmente unico, anche se la risorsa diventa non disponibile.
\end{description}
\subsection{Sintassi URL}
\begin{center}
    \color{blue}\textit{scheme://host:port/path}
\end{center}
\begin{itemize}
    \item scheme: protocollo di accesso
    \item host: nome di dominio o indirizzo IP
    \item port: numero di porta del servizio richiesto
    \item path: identifica la risorsa nel contesto sopra-descritto
\end{itemize}
Nelle URL HTTP si aggiunge anche un parametro \textit{\color{blue}?query} dopo \textit{\color{blue}path}.
Questa stringa viene interpretata dal server e serve a passare informazioni aggiuntive nella richiesta.
\subsection{URI Assolute e Relative}
\begin{description}
    \item[URI assoluta] identifica una risorsa indipendendentemente dal contesto.
    \item[URI relativa] identifica una risorsa in relazione ad un'altra URL e viene interpretata dal client prima di mandare la richiesta. 
\end{description}
\section{HyperText Transfer Protocol (HTTP)}
Protocollo {\color{blue} stateless} pubblicato nel 1990 e utilizzato fino ad ora nel World Wide Web.
Una caratteristica fondamentale dell'HTTP e' la tipizzazione e negoziazione della rappresentazione dei dati percio' non e' limitato all'ipertesto.

Il client stabilisce una connessione e invia richieste ({\color{blue}request}) tramite HTTP al server, il quale invia messaggi di risposta ({\color{blue}response})
Usa TCP dato che il trasferimento deve essere senza perdite e che i ritardi non sono \textit{un problema}

Dalla versione 1.1 la connessione client-server, a meno che diversamente espresso dal client, e' persistente; percio' il client puo' assumere che il server manterra' la connessione aperta.
Cio' porta a minor utilizzo di CPU e ritardi minori dato che non c'e' bisogno di riaprire una nuova connessione per ogni richiesta.
Inoltre si ha un mignor controllo di congestione.

La connessione viene chiusa dal server solo quando esplicitamente richiesto dal client nell'header del messaggio, o se non riceve piu' richieste (time out)

% \subsection{Pipelining e HeadOfLineBlocking}

% Per migliorare le le prestazioni il client puo' inviare contemporaneamente piu' richieste al server, il quale \textbf{DEVE} rispondere nello stesso ordine in cui sono state ricevute.

% Se pero' una richiesta richiede piu' tempo al server per essere processata blocca tutte le risposte successive e porta all'HOLB. In questi casi non si ha un miglioramento delle performance.

% \newpage
\subsection{HTTP Request}

La richesta e' formata da:
\begin{quote}
    Request-Line \\
    *( general-header \\
    $\vert$ request-header \\
    $\vert$ entity-header ) \\
    CRLF \\
    $[$ message-body $]$
\end{quote}

\subsection{HTTP Response}

La risposta e' formata da:
\begin{quote}
    Status-Line \\
    *( general-header \\
    $\vert$ response-header \\
    $\vert$ entity-header ) \\
    CRLF \\
    $[$ message-body $]$
\end{quote}

\paragraph{Request-Line}
\begin{quote}
    Method SP \\
    Request-URI SP \\
    HTTP-Version CRLF
\end{quote}
\paragraph{Status-Line}
\begin{quote}
    HTTP-Version SP \\
    Status-Code SP \\
    Reason-Phrase CRLF
\end{quote}
\paragraph{General-header} Relativi alla connessione.
\paragraph{Entity-header} Relativi all'entita' trasmessa.
\paragraph{Request-header} Relativi alla richiesta.
\paragraph{Response-header} Nel messaggio di risposta.
\subsection{Content Negotiation}
Le risorse possono essere disponibili in piu' rappresentazioni; Il client puo' richiederne una in particolare tramite richieste su Request e Entity headers.
\subsection{Request methods}
\paragraph{OPTIONS} Il server ritorna solo le opzioni di comunicazione associate ad una URL (capacita', metodi esposti, \dots).
\paragraph{GET} Richiede il trasferimento della risorsa indicata. Sono possibili i \textbf{conditional-get} (If-Mofidied-Since, If-Match, \dots).
\paragraph{HEAD} Simile alla GET ma il server non trasferisce il message-body. Utile per controllare lo stato dei documenti (validita', modifiche, \dots).
\paragraph{POST} Serve per inviare al server informazioni inserite nel body del messaggio.
\paragraph{PUT} Chiede al server di creare/modificare una risorsa (Recuperabile poi tramite GET).
\paragraph{DELETE} Il client chiede di eliminare una risorsa identidficata dalla Request-URI.
\subsection{Caching}
E' possibile memorizzare copie temporanee di risorse Web e ri-servirle al client senza contattare il server.
Si possono avere:
\begin{description}
    \item[User-Agent Cache] Risorsa mantenuta dal browser.
    \item[Proxy Cache] Il proxy mantiene una copia delle risorse richieste. Quando un browser richiede una risorsa cached essa gli viene riservita dal proxy senza interrogare nuovamente il server.
\end{description}
\subsection{Cookies}
Essendo stateless, non e' mantenere una memora su un'applicazione web (come ad esempio l'identita' di un client).
Una soluzione e' utilizzare i cookies, cioe' id unici che il client presenta al server ogni volta che esso effettua una richiesta.

Alla prima connessione, il server invia la normale risposta HTTP + una line \textbf{Set-cookie: id}.
Il client si salva il cookie e lo associa al server. Ogni successiva richiesta a quel server conterra' quel cookie.

\section{Versioni HTTP}
\paragraph{HTTP/1.1}
Permette di instaurare piu' connessioni contemporaneamente perciò il client puo' inviare piu' richieste al server contemporaneamente \bluetext{(Pipelining)}.

Il server deve pero' rispondere nello stesso ordine in cui sono arrivate le richieste perciò, se una richiesta richiede piu' tempo per essere elaborata, tutte le risposte successive non possono essere inviate \bluetext{(HeadOfLineBlocking)}.
\paragraph{HTTP/2}
Aggiunge caratteristiche allo standard HTTP come:
\begin{itemize}
    \item Multiplexing delle richieste su un'unica connessione TCP attraverso la definizione di:
    \begin{description}
        \item[Frame] Un messaggio HTTP viene mappato da una sequenza di Frame (come Data Frame, Headers Frame, \dots).
        \item[Stream] E' un flusso bidirezionale di frame all'interno di una connessione TCP; mediante l'astrazione di essi, e' possibile effettuare il multiplexing delle richieste (piu' stream su un'unica connessione). E' possibile anche associargli un peso (priorità), e una dipendenza verso altri stream.
    \end{description}
    \item Compressione delle intestazioni.
    \item Server Push (permette al server di inviare risorse aggiuntive oltre a quella richiesta dal client).
\end{itemize}
\paragraph{HTTP/3}
E' un'evoluzione di HTTP/2 che usa i servizi di QUIC, un protocollo di trasporto basato su UDP. QUIC aggiunge controllo del flusso e della congestione, e rilevazione dell'errore e delle perdite. Grazie all'astrazione dello stream a livello di trasporto, il rallentamento o la perdita di uno di essi non influisce sugli altri.
