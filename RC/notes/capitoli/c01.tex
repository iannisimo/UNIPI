\chapter{Introduzione alle Reti}

\section{Introduzione}

\paragraph{Rete}

Con rete si intende un'interconnessione di dispositivi in grado di scambiarsi informazioni.
Le reti sono commposte da elementi quali:
\begin{itemize}
    \item Sistemi terminali \textit{(Host)}
    \subitem Macchine degli utenti finali
    \subitem Server \textit{Fornitori di servizi}
    \item Switch: Dispositivi adibiti all'interconnessione locale di Host
    \item Router: Dispositivi di interconnessione di reti diverse
    \item Collegamenti: I mezzi tramite i quali vengono trasferite le informazioni
    \subitem Cavi in rame
    \subitem Fibra ottica
    \subitem Onde radio \textit{(WiFi)}
\end{itemize}

\section{Tipi di rete}
\paragraph{LAN}
Con \textit{LocalAreaNetwork} o \textit{Rete Locale} si intende un insieme di Host appartenenti allo stesso ente \textit{(Organizzazione, Casa, Scuola)} in grado di comunicare.\newline
Le LAN possono essere:
\begin{itemize}
    \item a cavo condiviso: Tutti gli host condividono lo stesso cavo per comunicare.
    Questo sistema non e' piu' usato perche' poco efficente.
    \item con switch: Tutti gli host sono collegati a uno switch che instrada le informazioni nella direzione desiderata.
    Questo sistema e' molto piu' efficiente perche' le macchine non hanno bisogno di monopolizzare la rete.
\end{itemize}
\paragraph{WAN}
Per \textit{WideAreaNetwork} o \textit{Rete Geografica} si intende una rete formata da piu' LAN e/o singoli host separati da grandi distanze.
Essa viene gestita da da un operatore che fornisce il servizio di interconnessione ai clienti.\newline
Le WAN si distinguono in: 
\begin{itemize}
    \item WAN punto-punto
    \item WAN a commutazione
\end{itemize}
Una applicazione tipica sono reti locali appartenenti ad un'azienda interconnesse tramite WAN p-p
\section{Tecniche di commutazione}
I due sistemi principali per determinare il percorso tra due host e dedicargli le risorse sono:
\begin{itemize}
    \item Circuit-switched network \textit{(Commutazione di circuto)}
    \item Packet-switched network \textit{(Commutazione di pacchetto)}
\end{itemize}
\paragraph{Commutazione di circuito}
Per la commutazione di circuito si procede instaurando un cammino dedicato tra i due host: vengono assegnate le risorse necessarie alla comunicazione e sono garantite per l'intera durata della connessione.
Cio' significa che una volta instaurata la connessione, essa non verra' disturbata in alcun modo.\newline
I principali problemi di questa tecnica sono pero' il tempo di instaurazione della connessione \textit{(risorse non disponibili)}, e il non sfruttamento delle risiorse disponibili durante i \textit{silenzi} nella comunicazione.
\paragraph{Commutazione di pacchetto}
Nelle connessioni a commutazione di pacchetto il flusso di dati viene diviso in pacchetti ed essi vengono \textit{spediti sulla rete} sul percorso prescelto.
Le risorse vengono quindi utilizzate solo se necessarie e possono essere condivise da pacchetti provenienti da host differenti.

Ogni nodo della rete si occupa di ricevere e riservire i pacchetti che gli arrivano. Per fare cio' il commutatore, dopo aver ricevuto un pacchetto, lo mette in una coda di tipo FIFO; quando e' pronto a ritrasmettere preleva il primo pacchetto dalla coda.
Cio' porta a dei ritardi (Il commutatore deve ricevere l'intero pacchetto per reinviarlo, i pacchetti potrebbero dover \textit{aspettare} in coda) e a delle perdite di pacchetti (coda piena).\newline
Questo metodo si chiama \textbf{Store and Forward}.
\section{Internet}
Con \textbf{i}nternet si intende un sistema formato da due o piu' reti comunicanti.
L'\textbf{I}nternet e' l'insieme di reti piu' comune. Ogni rete che intende aggiungersi ad essa deve seguire L'Internet Protocol (IP) e rispettare certe convenzioni.

L'infrastruttura di Internet fornisce servizi di comunicazione alle applicazioni
\begin{itemize}
    \item Senza connesione \textbf{(UDP)}
    \item Orientati alla connessione \textbf{(TCP)}
\end{itemize}

Sono stati definiti dei \textbf{protocolli} di comunicazione per le applicazioni piu' comuni di Internet \textit{(TCP, IP, HTTP, FTP...)}

Ci sono delle organizzazioni adibite alla definizione degli standard di internet:
\begin{itemize}
    \item \textbf{IETF} Internet Engineering Task Force
    \begin{itemize}
        \item Studia e sviluppa i protocolli in uso su internet.
        \item Pubblica i documenti ufficiali che li descrivono sotto forma di RFC/STD \textit{(Request For Comments, STanDards)}
    \end{itemize}
    \item \textbf{ICANN} Internet Corporation for Assigned Names and Numbers
    \begin{itemize}
        \item Coordina i DNS
        \item Assegna i gruppi di indirizzi di rete
        \item Ha funzioni di controllo semplice dello sviluppo di Internet
    \end{itemize}
    \item \textbf{W3C} World Wide Web Consortium
    \begin{itemize}
        \item Sviluppa di standard aperti \textit{(HTML, XML...)}
    \end{itemize}
\end{itemize}

\subsection{Strati della rete}
Le reti degli host si collegano a Internet tramite gli ISPs \textit{Internet Service Provider}.
I livelli della rete sono:
\begin{description}
    \item[Livello 3] ISP di accesso: Sono quelli a cui si connettono comunemente le reti locali.
    \item[Livello 2] ISP regionali: Sono dei collegamenti intermedi che uniscono tutti gli ISP di livello 2 in una zona geografica
    \item[Livello 1] Dorsali: Esse sono la parte piu' \textit{alta} di Internet, tutti gli altri ISP si connettono ad esse. \textit{(ne esistono circa 11)}
\end{description}

\subsection{Peering point}
Sono accordi tra due ISP che gli permettono di ricevere e riinoltrare il traffico da uno all'altro:
per fare cio' esistono gli IXP \textit{(Internet eXchange Point)} ovvero sistemi, anche gestiti da aziende di terzi, che effettuano il peering.
\subsection{Reti di acesso}
Il collegamento tra l'utente e Internet e' detto \textbf{rete di accesso}.
\begin{itemize}
    \item Accesso via rete telefonica
    \begin{itemize}
        \item dial-up
        \item Digital Subscriber Line \textbf{(DSL)}
        \item Fibra ottica
    \end{itemize}
    \item Accesso tramite reti wireless
    \begin{itemize}
        \item 3G, 4G, 5G
    \end{itemize}
    \item Collegamento diretto
    \begin{itemize}
        \item Collegamenti WAN dedicati per aziende, universita'...
    \end{itemize}
\end{itemize}
\newpage
\section{Metriche di riferimento}

\paragraph{Larghezza di banda (Bandwidth)}
Larghezza in Hertz dell'intervallo di frequenze utilizzato per la trasmissione.
\paragraph{Velocita' di trasmissione (bitrate)}
Quantita' di dati trasmissibili nell'unita' di tempo (bps).
\paragraph{Troughput}
Quantita' di dati trasmissibili da un nodo A ad un nodo B in una unita' di tempo.
Tiene di conto anche di perdite sulla rete, protocolli, ecc\dots
\paragraph{Latenza}
Tempo che intercorre tra l'invio e la ricezione del primo bit di un messaggio.
\begin{quote}
    \small Latenza = elaborazione + accodamento + trasmissione + propagazione
\end{quote}
\subsection{Ritardi}
\paragraph{Ritardo di elaborazione}
Causato dal sistema di controllo degli errori e dal sistema che determina il canale di uscita.
\paragraph{Ritardo di accodamento}
Tempo tra l'inserimento nella coda di trasmissione e la ritrasmissione del pacchetto.
\paragraph{Ritardo di trasmissione}
Tempo impiegato per trasmettere un pacchetto sul mezzo di trasmissione.
\begin{quote}
    Misurato in PacketLenght / BitRate
\end{quote}
\paragraph{Ritardo di propagazione}
Tempo che inpiega un bit ad essere propagato da un nodo all'altro.
\begin{quote}
    Misurato in LinkLenght / PropSpeed $(3-2*10^-8)$
\end{quote}
\paragraph{Ritardo end-to-end}
Ritardo cumulato tra tutti i nodi in una trasmissione. E' pari alla sommatoria dei ritardi tra i vari nodi del collegamento.
\subsection{Prodotto rate-ritardo}
Numero massimo di bit che possono \textit{essere contenuti} nel mezzo trasmissivo.
\newpage 
\section{Modelli stratificati}
Il modello statificato permette di scomporre un sistema complesso in piu' sistemi piu' facili da implementare e comprendere.
La modularizzazione dei livelli permette di dividere l'interfaccia dall'implementazione di un servizio.
Percio' dall'esterno ogni modulo e' visto come un'interfaccia che: accetta determinati parametri in ingresso, esegue le sue mansioni, e ritorna una risposta.
Quindi l'implementazione puo' essere cambiata senza che il resto del sistema \textit{ne venga a conoscenza}
\section{OSI RM \small(Open System Interconnection Reference Model)}
Nel 1976 sono iniziati i lavori per definire uno standard aperto per i protocolli di Internet.
La ISO ha da prima pubblicato questi standard sotto forma del OSI-RM, che poi e' diventato uno standar internazionale nel 1983 (ISO 7498).

Il modello ISO/OSI prevede la statificazione del protocollo di telecomunicazione.

\paragraph{Gli strati OSI sono:}
\begin{description}
    \item[7] Applicazione - elaborazione dati
    \item[6] Presentazione - unificazione dati 
    \item[5] Sessione - controllo del dialogo
    \item[4] Trasporto - trasferimento dati tra hosts
    \item[3] Rete - instradamento del traffico
    \item[2] Datalink - consegne trame sul link
    \item[1] Fisico - trasmette un flusso di bit
\end{description}
Le informazioni si propagano dal livello piu' alto (applicazione) fino al piu' basso per poi passare sul mezzo trasmissivo e risalire i livelli fino alla destinazione.

Ogni livello aggiunge all'informazione del livello superione una propria sezione informativa (header / trailer)
Questo processo di incapsulamento delle informazioni e' reversibile percio' ogni livello e' in grado di estrarre i dati degli strati superiori.

\section{Stack Protocollare TCP-IP}
E' la famiglia di protocolli attualmente utilizzata in Internet.
E' definita attualmente da cinque livelli:
\begin{description}
    \item[Applicazione] Applicazioni di rete, collegamento logico end-to-end, scambio di messaggi tra processi. \textit{\tiny ftp, smtp, http}
    \item[Trasporto] Trasferimento dati end-to-end. \textit{\tiny tcp, udp}
    \item[rete] Instradamento dei datagrammi. \textit{\tiny IP, ICMP}
    \item[Link] Trasferimento dei dati in frame tra elementi vicini. \textit{\tiny ppp, ethernet, \dots} 
    \item[Fisico] Trasferimento di bit di un frame sul mezzo trasmissivo. 
\end{description}