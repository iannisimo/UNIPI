\chapter{Lo Strato di Trasporto: TCP}
\section{Proprietà del servizio}
I processi effettuano un handshake {\tiny\color{red}COVID!1!}.
Lo strato della connessione risiede sui puni terminali e non sui nodi della rete.

Permette di trasferire un flusso continuo di dati in formato bidirezionale (full-duplex).
Inoltre consente di assegnare una connessione diretta processo - processo.
Offre anche controllo di connessione tramite meccanismi di inizio e fine trasmissione.
Tramite alcuni bit di controllo, permette di correggere alcuni tipi di errore.
Con il controllo di flusso si evita di spedire piu' dati di quanti il destinatario sia in grado di trattare.
Nel caso di sovraccarico della rete, tramite il controllo di congestione, si hanno sistemi di recupero della connessione.
\paragraph{Trasferimento bufferizzato}
Il protocollo TCP puo' suddividere il flusso di byte in segmenti. Per farlo e' necessario un buffer dove immagazzinare i dati finché non ce n'e' un numero sufficiente da essere spediti.
Cio' consente un minor numero di messaggi scambiati per trasferire una sequenza di byte.
\section{Formato dei segmenti}
\subsection{Numeri di sequenza e di riscontro}
\paragraph{Numero di sequenza} e' il numero del primo byte di un segmento.
Generalmente si parte da un Initial Sequence Number (ISN) casuale.
\paragraph{Numero di riscontro} e' il +1 del numero dell'ultimo byte ricevuto correttamente.

\textit{ACK=x} = significa: ho ricevuto tutti i byte fino a \textit{x-1}, aspetto \textit{x}.
\subsection{Forma dei segmenti}
L'header aggiunto al messaggio contiene:
\begin{description}
    \item[Porta sorgente] 16bit
    \item[Porta destinazione] 16bit
    \item[n-SEQ] Numero di sequenza, 32 bit
    \item[n-ACK] Numero di riscontro, 32 bit
    \item[HLEN] Lunghezza dell'header espressa in parole da 4Byte, 4bit
    \item[Reserved] 6bit
    \item[URG] Il campo puntatore contiene i dati da trasferire in via prioritaria, 1bit
    \item[ACK] Considerare n-ACK, 1bit
    \item[PSH] Trasferimento immediato da trasporto a applicazione
    \item[RST] Reset della connessione
    \item[SYN] Sincronizza n-SEQ
    \item[FIN] Chiusura della connessione   
    \item[Dimensione finestra] Numero di byte {\tiny a partire da n-ACK}, elaborabili, 16bit
    \item[Checksum] Considera l'intero pacchetto; serve a rilevare gli errori, 16bit
    \item[Puntatore urgente] {\small Punta al primo Byte non URG partendo da n-SEQ}, 16bit
    \item[Opzioni e riempimento] {\small Negoziazione di vari parametri opzionali}, 0-40Byte
\end{description}
\newpage
\section{Gestione della connessione}
\subsection{Handshake a tre vie}
\begin{description}
    \item[Richiesta di connessione] Il client invia una richiesta al server con:
    \begin{description}
        \item[SYN] 1
        \item[n-SEQ] \#rand {\tiny = x}
    \end{description}
    Il server inizializza due buffer e le variabili di connessione per il controllo di flusso e congestione.
    \item[Autorizzazzione connessione (SYNACK)] Il server risponde con:
    \begin{description}
        \item[SYN] 1
        \item[ACK] 1 
        \item[n-SEQ] \#rand {\tiny = y}
        \item[n-ACK] x+1
    \end{description}
    Il client inizializza gli stessi buffer e variabili.
    \item[ACK] Riscontro positivo dell'avvenuta connessione, contiene:
    \begin{description}
        \item[SYN] 0
        \item[ACK] 1
        \item[n-SEQ] x+1
        \item[n-ACK] y+1 
    \end{description}
    Connessione stabilita.
\end{description}
\subsection{Chiusura della connessione}
\setlength{\columnseprule}{0.4pt}
\setlength{\columnsep}{5em}
\begin{multicols}{2}
    \begin{description}
        \item[C$\implies$S] Chiusura da parte\\del client:
        \begin{description}
            \item[FIN] 1
            \item[n-SEQ] x  
        \end{description} 
        Il client non puo'\\piu' inviare dati.
        \item[S$\implies$C] ACK di chiusura:
        \begin{description}
            \item[ACK] 1
            \item[n-ACK] x+1  
        \end{description}  
        Il server puo' ancora\\inviare dati, il client no.
        \item[S$\implies$C] Chiusura da parte\\del server:
        \begin{description}
            \item[FIN] 1
            \item[n-SEQ] y  
        \end{description} 
        Il server aspetta la\\ricevuta di chiusura.
        \item[C$\implies$S] ACK di chiusura.
        \begin{description}
            \item[ACK] 1
            \item[n-ACK] y+1  
        \end{description} 
        Connessione terminata.
    \end{description}
\end{multicols}
\setlength{\columnseprule}{0pt}
\setlength{\columnsep}{1em}
Quando il client riceve FIN dal server entra nello stato {\color{blue} TIME\_WAIT} e ci resta per un tempo dettato da MSL (diverso su macchine differenti) prima di poter chiudere totalmente la connessione. Questo serve nel caso l'ultimo ACK venga perso perché a un certo punto uno dei due host richiederà la chiusura passiva e l'altro dovrà rispondere con un ACK.
Allo stesso modo dell'handshake di apertura, si puo' avere un three-way handshake di chiusura.
\section{Stati TCP}
\subsection{Stati client}
\begin{description}
    \item[SYN-SENT] Dopo aver inviato una richiesta di connessione, si aspetta la conferma.
    \item[ESTABLISHED] La connessione e' pienamente stabilita, e' possibile trasferire dati.
    \item[FIN-WAIT-1] Il client aspetta una richiesta di chiusura o l'ack di una sua richiesta di terminazione.
    \item[FIN-WAIT-2] Aspetta la richiesta di terminazione di connessione da un host remoto. 
    \item[TIME-WAIT] Vedi sopra.
    \item[CLOSED] Connessione chiusa. Non e' piu' possibile comunicare.
\end{description}
\subsection{Stati server}
\begin{description}
    \item[LISTEN] In attesa di una connessione TCP qualunque.
    \item[SYN-RECEIVED] Si entra in questo stato appena si riceve una richiesta di connessione.
    \item[ESTABLISHED] Vedi sopra.
    \item[CLOSE-WAIT] Si aspetta la richiesta di chiusura dal client.
    \item[LAST-ACK] Si aspetta l'ack di chiusura.
    \item[CLOSED] Vedi sopra.
\end{description}
\newpage
\section{Trasferimento di dati affidabile}
\subsection{Esempio Telnet su TCP}
\begin{description}
    \item[C$\implies$S] L'utente digita C
    \begin{description}
        \item[n-SEQ] 42
        \item[n-ACK] 79
        \item[data] "C"   
    \end{description}
    \item[S$\implies$C] ACK per ricevuta di C
    \begin{description}
        \item[n-SEQ] 79
        \item[n-ACK] 43
        \item[data] "C"   
    \end{description}
    \item[S$\implies$C] ACK dell'ACK
    \begin{description}
        \item[n-SEQ] 43
        \item[n-ACK] 80
    \end{description}
\end{description}
Il mittente puo' anche inviare piu' segmenti senza attendere il riscontro (pipelining).
\subsection{Eventi lato mittente}
Lo strato di trasporto riceve i dati dall'applicazione, li incapsula, gia assegna un numero di sequenza, ed avvia un timer di ritrasmissione (RTO).

La ritrasmissione avviene in caso di:
\begin{description}
    \item[Timeout] Non si riceve un ACK prima della scadenza del RTO.
    \item[ACK duplicato] Il mittente riceve tre ACK uguali significa che il segmento successivo a quello e' andato perso. Si ha quindi la {\color{blue}fast retransmission}.  
\end{description}
\subsection{Segmenti fuori sequenza}
Se i dati arrivano fuori sequenza (ordinamento sparso), l'entità TCP destinataria puo' memorizzarli temporaneamente ma il protocollo non specifica che utilizzo farne.

Nelle versioni piu' recenti si implementa SACK, che manda l'ACK dei pacchetti fuori sequenza in OPTIONS.
\subsection{Eventi lato destinatario}
Tutti i segmenti inviati per trasmettere dati includono ACK.
\paragraph{Delayed ACK} Se il destinatario riceve un segmento atteso, puo' ritardare l'invio di ACK di 500ms a meno che non riceva un altro segmento.
\paragraph{Fast retrasmission request}Se il destinatario riceve un segmento fuori sequenza, un duplicato, o un capisce che c'e' un segmento mancante, risponde subito con un ACK indicando il segmento da cui ripartire.
\subsection{Calcolo del timeout}
Il RTO deve essere maggiore del Round Trip Time (RTT) (tempo tra l'invio di un messaggio e la ricevuta dell'ACK).
\begin{quote}
    $eRTT = (1 - \alpha) * eRTT + \alpha * sRTT$
\end{quote}
\begin{description}
    \item[eRTT] RTT stimato, cumulativo su piu' misure.
    \item[sRTT] RTT di un segmento trasmesso con successo.
    \item[\textit{alpha}] $1/8$ (RFC2988)
\end{description}
\begin{quote}
    \small$dRTT = (1 - \beta) * dRTT + \beta * (sRTT - eRTT)$
\end{quote}
\begin{description}
    \item[dRTT] Deviazione di eRTT da sRTT
    \item[\textit{beta}] $1/4$ (RFC2988)
\end{description}
\begin{quote}
    {\boldmath$RTO$} $ = eRTT + 4*dRTT$
\end{quote}
In molte implementazioni dopo un errore si raddoppia RTO
\subsection{Finestra di trasmissione}
Il buffer di invio, si dividono i byte in {\color{blue}finestre}.
La finestra di trasmissione e' la sottosezione del buffer di invio che deve essere spedita con il prossimo messaggio.
Ha dimensione variabile (modificata in base alla condizione della rete) per permettere i controlli di flusso e congestione.
Viene fatta avanzare non appena si riceve l'ACK dell'ultima trasmissione.
\section{Controllo di flusso}
Il destinatario di un messaggio mantiene un buffer di ricezione.
il processo sullo strato applicazione legge i dati da una finestra di ricezione (non necessariamente quando arrivano).
Si intende con {\color{blue}controllo di flusso} la capacità del mittente di evitare di saturare il buffer del destinatario.
Il mittente mantiene una variabile detta {\color{blue}finestra di ricezione} (rwnd) che detta lo spazio disponibile nel buffer ricevente; essa e' passata dal destinatario al mittente tramite il campo window nell'header TCP.
\begin{quote}
    $rwnd = RcvBuffer - (LastByteReceived - LastByteRead)$
\end{quote}
\newpage
Il mittente si assicura che:
\begin{quote}
    $LastByteSent - LastByteAcked < rwnd$
\end{quote}
\begin{quote}
    Se $rwnd = 0$ il mittente lo \textit{richiede} con un segmento sonda da un byte
\end{quote}
\section{Controllo di congestione}
Se si provano a spedire troppi dati su una rete che non puo' supportarli avviene il fenomeno della congestione; cio' puo' provocare:
\begin{itemize}
    \item Lunghi ritardi (accodamenti nei buffer)
    \item Perdita di pacchetti (overflow dei buffer)
\end{itemize}
TCP permette il {\color{blue}controllo di congestione}: se si percepisce uno scarso traffico, viene aumentata la frequenza di invio; altrimenti diminuisce.
\subsection{Algoritmo di controllo}
L'algoritmo utilizzato per regolare la frequenza di invio in funzione della congestione e' costituito da tre fasi:
\begin{itemize}
    \item Slow start
    \item AIMD 
    \item Fast recovery
\end{itemize}
\subsection{Finestra di congestione (Congestion Window)}
Misurata in Max Segment Size (MSS); E' il numero di segmenti che si possono inviare insieme, senza bisogno di aspettarne l'ACK. %FIXED: Ricontrolla sta roba cretino, tanto non hai capito sicuramente un cazzo;
\subsection{Slow start}
All'inizio la congestion window (CW) e' posta a 1MSS.
Per ogni ACK, CW viene aumentata di 1MSS. Cio' significa che essa raddoppia per ogni RTT.
\subsection{Additive Increase Multiplicative Decrease}
La CW viene aumentata in modo da avere un incremento pari a 1MSS per ogni RTT;
Cio' significa che per ogni ACK, cwnd = cwnd+1/cwnd.

Ad ogni evento di perdita di ACK la CWND viene dimezzata.
\subsection{TCP RENO}
Viene definita una variabile di threshold alla quale viene assegnato un valore alto;
Finché $cwnd < threshold$, siamo in slow start (cwnd aumenta esponenzialmente).\newline
Non appena $cwnd > theshold$, si entra in AI.
\paragraph{Se ricevo 3 ACK duplicati} entro in fast recovery:
\[ threshold = \dfrac{cwnd}{2} \]
\[ cwnd = threshold + 3MSS \]
Finché continuano ad arrivare ACK duplicati
\[ cwnd = cwnd + 1 \]
Non appena arriva un ACK non duplicato
\[ cwnd = threshold \]
\paragraph{Se perdo un ACK per timeout} entro in slow start:
\[ threshold = \dfrac{cwnd}{2} \]
\[ cwnd = 1MSS \]
\subsection{TCP Tahoe}
Molto simile a RENO ma timeout e 3 ACK duplicati vengono gestiti allo stesso modo (si va in slow start)
\subsection{Throughput}
Indicando con $W$ la dimensione massima raggiunta dalla finestra, il throughput medio e'
\[ Throughput = \dfrac{0,75 \times W}{RTT} \]
\subsection{Equità (fairness)}
In condizioni perfette, $K$ connessioni TCP che passano sullo stesso link di capacita' $R$ bit/s, le connessioni hanno gli stessi valori di $MSS$ e $RTT$; in questo caso ogni connessione trasmette $R/K$ bit/s.
In condizioni reali pero' le connessioni con $RTT$ piu' piccolo (migliori) variano piu' velocemente $congwin$ e raggiungono throughput superiori.