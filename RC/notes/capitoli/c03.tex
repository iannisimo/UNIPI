\chapter{Lo Strato Applicativo: Telnet}
TErminaL NETwork e' un protocollo che permette l'uso di shell su macchine remote.
Permette al client di effettuare una sessione di login al server, dopo la quale esso puo' accedere a comandi e programmi disponibili sulla macchina remota.

Dopo il login, vengono passate le battute dei tasti al server come standard-input e l'output viene riportato direttamente al client.

Il protocollo TELNET [RFC854] detta che:
\begin{itemize}
    \item Il client stabilisce una connessione di tipo TCP (tipicamente sulla porta 23) con il server. la connessione persiste per tutta la durata della sessione di login.
    \item Vengono \textit{"collegate"} le STDIN e STDOUT dei tue terminali.
\end{itemize}
Il server Telnet utilizza uno Pseudo Terminal Driver per eseguire i comandi.
\section{Network Virtual Terminal: NVT}
Telnet deve poter operare con il numero massimo di sistemi quindi deve poter lavorare con client su sistemi operativi diversi.

Per fare questo il client e il server passano attraverso un terminale virtuale in modo da avere una sola codifica ed entrambi effettuano una conversione dalla propria a quella del NVT.
Sulla rete vengono quindi trasferiti i comandi con l'\textbf{NVT Character Set}.
\newpage
\section{NVT Character Set}
I dati vengono trasferiti tramite 7-bit US-ASCII.
\begin{itemize}
    \item Ogni carattere e' inviato come un byte con il primo bit settato a 0.
    \item Per le sequenze di comandi si imposta il bit piu' significativo a 1.
    \item I comandi come EOL iniziano con 0xFF (Interpret As Command \textbf{IET})
\end{itemize}
