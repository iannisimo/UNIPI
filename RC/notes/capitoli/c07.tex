\chapter{Lo Strato di Trasporto: UDP}
I datagrammi UDP possono essere perduti o consegnati fuori sequenza, perciò i processi devono inviare messaggi di dimensione limitate, e devono essere contenuti in un solo datagramma.
\subsection{Differenze tra TCP e UDP}
\begin{tabular}{|c|c|}
    \hline
    TCP & UDP\\
    \hline
    Connection-oriented & Connection-less\\
    Mandatory checksum & Optional checksum\\
    Introduce ritardi & Nessun ritardo aggiunto\\
    Aggiunge informazioni & Datagramma minimale\\
    \hline
\end{tabular}
\subsection{Formato dei datagrammi}
\begin{description}
    \item[Porta sorgente] 16bit
    \item[Porta destinazione] 16bit
    \item[Lunghezza] Lunghezza totale del segmento, 16bit
    \item[Checksum] Controllo errore end-to-end, opzionale, 16bit
    \item[Data] max 65535Byte - 8Byte intestazione
\end{description}
\subsection{Utilizzi UDP}
\begin{itemize}
    \item Processi che richiedono scambio di dati con volume limitato e senza controlli di flusso/errori.
    \item Processi con meccanismi interni di controllo di flusso/errore.
    \item Trasmissioni multicast (piu' destinatari).
    \item Applicazioni interattive che non tollerano ritardi variabili.
\end{itemize}
