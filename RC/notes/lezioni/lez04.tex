\chapter{Lo Strato Applicativo: DNS}
\section{Introduzione}

Ogni rete e' identificata sulla rete tramite un indirizzo IP.
Il problema di essi e' che non sono facili da ricordare \textit{\tiny e sono brutti}.
E' quindi nato il bisogno di associare gli IP a dei nomi logici (e spesso mnemonici).
Per associare gli IP ai nomi e' stato creato il DNS.

Il DNS adotta il paradigma client-server.
Si affida al protocollo di trasporto sottostante per trasferire i messaggi.
E' costituito da:
\begin{itemize}
    \item Uno schema di assegnazione dei nomi
    \item Un database distribuito contenente le associazioni 
    \begin{center}
        $nome dominio \implies IP$
    \end{center}
    \item Un protocollo per la distribuzione delle informazioni sui nomi tra i name server
    \begin{itemize}
        \item Utilizza UDP (porta 53) [oppure TCP]
    \end{itemize}
\end{itemize}

\section{Servizi DNS}
\paragraph{Risoluzione} Traduzione $hostname \implies indirizzo IP$
\paragraph{Host aliasing} Traduzione $nomi \implies nome canonico / IP$
\paragraph{Mail server aliasing} Stessa cosa degli host, permette tra le altre cose di usare nomi identici per mail e web server.
\paragraph{Distribuzione di carico} Ad un hostname canonico possono corrispondere piu' indirizzi IP; il DNS restituisce la lista di IP variandone l'ordinamento ogni volta.
\newpage
\section{Spazio dei nomi}
