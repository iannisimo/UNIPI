\chapter{Dimostrazioni per induzione}

\section{Notazione per gli insiemi}

\begin{itemize}
	\item Naturali: $\N = \{0, 1, 2 ...\}$ 
	\item Interi: $\Z = \{..., -2, 1, ... \}$.
	\item Razionali: $\Q$ = razionali $\frac{m}{n}$ con  $n\ne 0$
	\item Reali: $\R$. Includono oltre ai razionali anche $\sqrt{2}=1,4142...$ $\pi$
	\item Complessi: $\mathbb{C}$ includono unità immaginaria $i =\sqrt{-1}$ e tutto ciò che si ottiene dai reali ed $i$ con somme e prodotti come $3 + 4i$
	\item Il modulo: $\Z /(12)$ interi modulo 12. Interi in base 12 come le ore dove 12 = 0.
\end{itemize}


\section{Relazioni d'ordine}

Un { \bf ordine totale} è un insieme su cui è definita una relazione binaria $\le$ che verifica le seguenti proprietà:

\begin{description}

\item [Transitiva] $x \le y \land y \le z \implies x \le z$ 
\item [Riflessiva] $x \le x$ 
\item [Antisimmetrica] $x \le y \land y \le x \implies x = y$
\item [Totalità] $x \le y \lor y \le x$ 
	
\end{description}

Un ordine parziale è un ordine dove non vale la {\bf totalità}
L'ordinamento sui numeri $\mathbb{N, Q, R}$ è totale.
Un ordine parziale è dato dalla relazione "x divide y" su $\N$


\subsection{Minore Stretto}

\paragraph{Definizione}

Data una relazione d'ordine $\le$ definiamo il corrispondente ordine stretto:

$$x < y \equiv x \le y \land x \ne y$$

Valgono le seguenti proprietà:

\begin{description}
	\item [Transitiva] $x < y \land y < z \equiv x < z$	
	\item [Antiriflessiva] $x \not < y$ 
	\item [Totalità] $x < y \lor y < x \lor x = y$ se l'ordine $\le$ è totale 
\end{description}

Definiamo il minore uguale quindi come: $x < y \equiv x < y \lor x = y$

\subsection{Buoni ordini}

Un insieme totalmente ordinato è un {\bf buon ordine} (insieme bene ordinato) se non ammette successioni infinite decrescenti: $a_0 > a_1 > a_2 > ...$

Un buon ordine è $\mathbb{N}$. Ogni successione decrescente termina dopo un numero di passi finiti.

$\mathbb{R}$ non è buon ordine.

I buoni ordini sono alla base delle definizioni per induzione e dimostrazioni per induzione


\subsection{Principio del minimo}

Dato un sottoinsieme $A$ di un insieme ordinato $X$, scriviamo $m = min(A)$ se valgono le condizioni seguenti:

\begin{itemize}

\item $m \in A$
\item $(\forall a \in A . m \le a)$

\end{itemize}

Ogni sottoinsieme non vuoto $A$ di $\mathbb{N}$ ha un minimo.
Un insieme è un buon ordine se qualunque sottoinsieme non vuoto ha un minimo


\section{Definizioni ricorsive}

Una funzione $f : \mathbb{N}$  si dice { \bf definita ricorsivamente } se è dato un valore iniziale per $f(0)$ e una legge per ottenere $f(n+1)$ a partire da $f(n)$

Un esempio è il fattoriale $n!$

$$\begin{cases}
0! = 1 \\
(n + 1)! = (n + 1) \cdot n!
\end{cases}$$

Applico la definizione ricorsiva

$$3! = 3 \cdot 2! = 3 \cdot 2 \cdot 1! = 3 \cdot 2 \cdot 1 \cdot 0! = 3 \cdot 2 \cdot 1 \cdot 1 = 6 $$



\subsection{La sommatoria}

Una somma di più addendi si indica con $\sum$ applicato ad un'espressione dotata di un indice (valore iniziale e finale). Ad esempio:

$$\displaystyle \sum_{i=2}^{5} {x_i} = x_2 + x_3 + x_4 + x_5$$ 

oppure

$$\displaystyle \sum_{j=1}^{4} {x_{j+1}} = x_2 + x_3 + x_4 + x_5$$

Possiamo definire la sommatoria $\sum_{i=1}^{n}{x_i}$ per ricorsione su $n$ come:

$$\begin{cases}
\displaystyle \sum_{i=0}^0 {x_i} = x_0 \\
\displaystyle \sum_{i=0}^{n+1} {x_i} = x_{n+1} + \sum_{i=0}^{n} {x_i}
\end{cases}$$


Esempio: la somma dei primi 5 quadrati:

$$\displaystyle \sum_{i=1}^{5} {i^2} = 1^2 + 2^2 + 3^2 + 4^2 + 5^2 $$

Somma dei primi $n$ quadrati:

$$\displaystyle \sum_{i=1}^{n} {i^2} = 1^2 + 2^2 + ... + n^2$$

La notazione $\sum$ è più precisa di quella con i puntini di sospensione.
I puntini sono incomprensibili per la macchina e possono rendere la formula ambigua.

\subsection{Linearità della sommatoria}

Applicando le proprietà distributive si verifica che un fattore comune $c$ si può portare fuori dalla sommatoria:

$$ \displaystyle \sum_{i=a}^{a+n} {(c \cdot x_i)} = c \cdot \sum_{i=a}^{a+n} {x_i}  $$

Analogamente 

$$ \displaystyle \sum_{i=a}^{a+n} {(x_i + y_i)} = \sum_{i=a}^{a+n} {x_i} + \sum_{i=a}^{a+n} {y_i} $$


\subsection{Produttoria}

La produttoria $\prod$ è definita analogamente a $\sum$ salvo che invece delle somme si fanno i prodotti.

$$\begin{cases}
\displaystyle \prod_{i=0}^0 {x_i} = x_0 \\
\displaystyle \prod_{i=0}^{n+1} {x_i} = \prod_{i=0}^{n}{x_i}  \cdot x_{n+1} 
\end{cases}$$

Se tutti i fattori $x_i$ sono uguali ad $x$ si ottiene $x^{n+1}$



\subsection{Fibonacci: Ricorsione sui valori precedenti}

Consideriamo la successione di Fibonacci $F_n (n \in \mathbb{N})$

$$\begin{cases}
F_0 = 0 \\
F_1 = 1 \\
F_{n+1} = F_{n+1} + F_n \\
\end{cases} $$


\section{Dimostrazioni per induzione}

Le dimostrazioni per induzione sono simili alle definizioni ricorsive. Le definizioni ricorsive servono per { \bf definire} delle funzioni su $\mathbb{N}$ mentre le {\bf dimostrazioni per induzione} servono a dimostrare degli enunciati (su $\mathbb{N}$)

Supponiamo che qualcuno abbia colorato i numeri naturali:

\begin{itemize}

\item $Red(7)$ 7 è rosso
\item $(\forall k \in \mathbb{N} . Red(k) \implies Red(k+1))$ Il successore di un numero rosso è rosso.

\end{itemize}


\subsection{Principio di induzione}

Consideriamo una proposizione $P(n)$ che dipende da un parametro $n \in \N$ come $3^n \leq n!$ oppure $n^3 \leq 2^n$. 

Sia $k \in \N$ un valore iniziale, supponiamo di dimostrare le due cose.

\begin{description}
\item [Base dell'induzione] $P(k)$ è $T$
\item [Passo induttivo] Per qualsiasi $n \geq k$ se vale $P(n)$ allora vale anche $P(n+1)$
\end{description}



Esercizio:

$$\displaystyle \sum_{i=1}^{n}{i} = \frac{n(n+1)}{2}$$

Nel caso $n = 1$ abbiamo $\displaystyle \sum_{i=1}^{i}{i} = 1 = \frac{1(1+1)}{2}$

Suppongo che 
$$\displaystyle \sum_{i=1}^{k}{i} = \frac{k(k+1)}{2}$$

per un certo $k$ (ipotesi induttiva).

Verifico il caso $k + 1$

$$
\displaystyle \sum_{i=1}^{k+1}{i} = \displaystyle \sum_{i=1}^{k}{i} + (k+1) 
$$ 

$\equiv$ (per definizione di $\sum$)

$$ \displaystyle \frac{k(k+1)}{2} + (k+1)$$

$\equiv$ (sfruttando l'ipotesi induttiva)

$$\displaystyle \frac{(k+1)(k+2)}{2}$$


\section{Induzione Forte}

Per dimostrare $\left(\forall n \in \mathbb{N} . P(n) \right)$ è sufficiente riuscire a dimostrare $P(0)$ (caso base) e che per ogni $n > 0$ vale l'implicazione $P(0) \land ... \land P(n - 1) \implies P(n)$ ({\bf passo induttivo})

Si può partire da qualsiasi valore iniziale. Nel dubbio fra induzione debole e forte, usiamo la forte. Se funziona la debole funziona anche la forte.

\subsection{Esempio}

Un intero $p > 1$ è primo se è divisibile solo per $1$ e per $p$, ovvero non si può scomporre come prodotto $p = ab$ con entrambi i fattori $a,b$ minori di $p$.

Facciamo un esempio di induzione forte nella dimostrazione del seguente teorema.

\paragraph{Teorema} Ogni numero $n > 1$ si scrive come prodotto di fattori primi

Assumiamo per ipotesi induttiva che sia vero per tutti gli interi $x$ dove $1 < x < n$. 

\begin{itemize}
	
	\item Dato $n > 1$ distinguiamo due casi
	\item Se $n$ è primo abbiamo fininto
	\item Se $n$ non è primo si scrive come prodotto $n = ab$ di due fattori $< n$
	
\end{itemize}
