\chapter{Teoria della Complessità}
\section{Problemi}
\subsection{Legenda}
\begin{description}
    \item[$\mathbf{\Pi}$] Problema
    \item[$\mathbf{I}$] Insieme delle istanze in ingresso
    \item[$\mathbf{S}$] Insieme delle soluzioni  
\end{description}
\subsection{Tipologie di problemi}
\paragraph{Problemi decisionali}
\begin{itemize}
    \item $S = \{0, 1\}$
    \item Istanze positive: $x \in I$ t.c. $\Pi(x) = 1$
    \item Istanze negative: $x \in I$ t.c. $\Pi(x) = 0$
\end{itemize}
\paragraph{Problemi di ricerca}
\begin{itemize}
    \item $S$ \textit{"libera"}
    \item Trovare \textit{una} soluzione al problema.
\end{itemize}
\paragraph{Problemi di ottimizzazione}
\begin{itemize}
    \item $S$ \textit{"libera"}
    \item Trovare la \bluetext{miglior} soluzione $s \in S$
\end{itemize}
I problemi di interesse pratico sono spesso di ottimizzazione. E' possibile pero' esprimerli sotto forma di problemi decisionali:
\begin{itemize}
    \item \bluetext{MAX-CLIQUE(G):} Richiede di trovare la CLIQUE piu' grande in un grafo G.
    \item \bluetext{CLIQUE(G, k):} Chiede se esiste una clique in G di almeno k vertici; non e' piu' difficile di \bluetext{MAX-CLIQUE}.
\end{itemize}
\section{Classi}
\subsection{Classi di Complessità}
Dati $\Pi$ e $A$, diciamo che $A$ risolve $\Pi$ se: $\exists x \in I$ t.c. $A(x) \and \Pi(x) = true$.
\subsection{Classi Time e Space}
\begin{description}
    \item[$Time(f(n))$] e' l'insieme dei problemi decisionali risolvibili in tempo $O(f(n))$
    \item[$Space(f(n))$] e' l'insieme dei problemi decisionali risolvibili in spazio $O(f(n))$
\end{description}
\subsection{Classe P}
E' la classe dei problemi risolvibili in tempo polinomiale \textit{($O(n^c)$, $c$ costante, $n$ dati in ingresso)}
\subsection{Classe PSpace}
E' la classe dei problemi risolvibili in spazio polinomiale \textit{($O(n^c)$, $c$ costante, $n$ dati in ingresso)}
\subsection{Classe ExpTime}
E' la classe dei problemi risolvibili in tempo esponenziale \textit{($O(c^n)$, $c$ costante, $n$ dati in ingresso)}
\subsection{Relazioni tra le classi}
\begin{description}
    \item[$P \subseteq PSpace$] Un algoritmo polinomiale ha accesso al piu' ad un numero polinomiale di locazioni.
    \item[$PSpace \subseteq ExpTime$]
\end{description}
% Ma porca puttana, non devo ridare ALGO, io gli algoritmi non li riscrivo.
\section{Certificati}
Un certificato e' un attestato di esistenza della soluzione a un problema. Si definisce solamente per istanze accettabili.
% PAG 17 
\subsection{Verifica}
Un problema $\mathbf{\Pi}$ e' verificabile it tempo polinomiale se:
\begin{itemize}
    \item $x \in I$ of $len = n$ ammette un certificato $y$ of $len$ polinomiale in $n$.
    \item Esiste un algoritmo di verifica che, applicato alle coppie $<x,y>$ permette di attestare che $x$ e' accettabile.
\end{itemize}
\section{Classe NP}
E' la classe dei problemi risolvibili in tempo polinomiale non deterministico.
Cio' significa che sono verificabili in tempo polinomiale.
Se si ha una soluzione si puo' verificare la sua legittimità in tempo polinomiale; Altrimenti la si puo' individuare con una ricerca esaustiva in tempo esponenziale.
\subsection{P e NP}
Sappiamo per certo che
\begin{quote}
    $P \subset NP$
\end{quote}
Ma non e' stato matematicamente dimostrata la congettura
\begin{quote}
    $P \not\equiv NP$
\end{quote}
\subsection{NP-Completo}
I problemi NP-Completi sono i problemi piu' \textit{"difficili"} all'interno della classe NP. Tutti i problemi NP sono riducibili in tempo polinomiale a problemi NP-Completi.
\paragraph{Riduzioni polinomiali} Dati i problemi $\Pi_1, \Pi_2$ e le rispettive istanze di input $I_1, I_2$, si dice che $\Pi_1$ si riduce in tempo polinomiale a $\Pi_2$ se esiste una funzione $f: I_1 \rightarrow I_2$ t.c. per ogni istanza x di $\Pi_1$
\begin{center}
    $x$ e' istanza accettabile di $\Pi_1$\\
    $\equiv$\\
    $f(x)$ e' istanza accettabile di $\Pi_2$
\end{center}
\subsection{NP-Arduo}
% TODO: Che cazzo sono i problemi NP-Ardui dio merda!@!
\subsection{Teorema di Cook}
Cook ha dimostrato che:
\begin{quote}
    Dato un problema $\Pi$ in NP e una qualunque istanza $x$ per $\Pi$\\
    Si puo' esprimere $\Pi$ sotto forma di espressione booleana in forma normale, la quale restituisce $true$ IIF x e' accettabile per $\Pi$
\end{quote}
\subsection{Gerarchia delle classi}
\begin{center}
    $(P \cup NP-Completi) \subseteq NP \subseteq PSpace \subseteq Exp \subseteq Decidibili$
\end{center}
\subsection{co-P \& co-NP}
% TODO: I cazzo di complementari, non ho voglia