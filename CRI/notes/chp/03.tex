\chapter{Teoria della Calcolabilità}
% Studia la potenza e le limitazioni dei sistemi di calcolo.
\section{Problemi computazionali}
\paragraph{Sono classificati in:}
\begin{itemize}
    \item Non decidibili
    \item Decidibili
    \begin{itemize}
        \item Trattabili \textit{(polinomiali)}
        \item Intrattabili \textit{(esponenziali)}
    \end{itemize}
\end{itemize}
\subsection{Calcolabilità e Complessità}
\paragraph{Calcolabilità} E' lo studio delle nozioni di algoritmo e di problema non decidibile.
\paragraph{Complessità} E' lo studio di algoritmi efficienti e di problemi intrattabili.
% Se mi viene voglia metto la dimostrazione dell'esistenza di problemi indecidibili
\subsection{Problema della rappresentazione}
\paragraph{Algoritmo}
Sequenza finita di operazioni, completamente e univocamente determinate.
Gli algoritmi possono essere formulati con modelli diversi come: modello matematico, algoritmo in pseudocodice, programma eseguibile\dots
Qualunque modello venga scelto, gli algoritmi devono essere descritti perciò \redtext{sono possibilmente \textbf{infiniti} ma \textbf{numerabili}}.
\paragraph{Problemi computazionali}
Sono funzioni matematiche che associano ad ogni insieme di input un risultato; \redtext{\textbf{non} sono numerabili}.
Cio' significa che:
\begin{center}
    \redtext{\#\{Problemi\} $>>$ \#\{Algoritmi\}}
\end{center}
Non esiste quindi un algoritmo di calcolo per ogni problema.
\paragraph{Il problema dell'arresto}
E' la dimostrazione data da Turing nel 1930 dell'esistenza di problemi non decidibili.

Prendiamo in considerazione il generico algoritmo
\begin{quote}
    $A: \{I\} \rightarrow \{0, 1\}$
\end{quote}
Che, in base a $I$ puo' terminare o non terminare.
Adesso poniamo per assurdo che esista un altro algoritmo $Arresto(A, D)$ che in tempo finito ritorna:
\begin{description}
    \item[true] se $A(D)$ termina
    \item[false] se $A(D)$ non termina 
\end{description}
$Arresto$ non puo' semplicemente simulare il comportamento di $A(D)$ perché se esso non terminasse, non terminerebbe neanche \bluetext{$Arresto$}.
    Se l'algoritmo \bluetext{$Arresto$} esistesse, esisterebbe anche l'algoritmo \bluetext{$Paradosso$} definito come:
\begin{algorithm}
    \While{Arresto(A, A)}{
        \;
    }
    \caption{Paradosso(A)}
\end{algorithm}\\
Se provo ad eseguire $Paradosso(Paradosso)$
\begin{quote}
    $Paradosso(Paradosso)$ termina\\
    $Arresto(Paradosso, Paradosso) = 0$\\
    $Paradosso(Paradosso)$ non termina
    \redtext{!!! ERR !!!}
\end{quote}
Cio' significa che l'algoritmo $Paradosso$ \redtext{\textbf{non puo' esistere}} (quindi neanche $Arresto$)
\section{Tesi di Church-Turing}
Tutti i \textit{(ragionevoli)} modelli di calcolo risolvono la stessa classe di problemi; perciò la decidibilità e' una proprietà del problema e non del modello utilizzato.