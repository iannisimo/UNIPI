\chapter{Rappresentazione matematica di oggetti}
\section{Definizioni}
\paragraph{Alfabeto} Per alfabeto si intende un insieme finito di caratteri o simboli.
\paragraph{Oggetto} Un oggetto e' una sequenza ordinata di elementi dell'alfabeto.
\section{Alfabeti e sequenze}
Considero l'alfabeto \redtext{$\Gamma$} con \redtext{$N$} oggetti da rappresentare.
Si considera \redtext{$s = |\Gamma|$} la cardinalità di $\Gamma$.
\begin{quote}
    Con \redtext{$d(s, N)$} si intende la lunghezza della sequenza piu' lunga .
\end{quote}
\begin{tabular}{l l}
    \textbf{$d(s, N)$} & E' la lunghezza della sequenza piu' lunga della rappresentazione scelta.\\
    \textbf{$d_{min}(s, N)$} & Valore minimo di $d(s, N)$ tra tutte le rappresentazioni possibili.
\end{tabular}
Tanto piu' si avvicina $d_{min}$ a $d$, tanto e' migliore la rappresentazione.
\subsection{Rappresentazione binaria}
\paragraph{s = 2, $\mathbf{\Gamma}$ = {0, 1}}
\begin{tabular}{l l}
    $2^k$ & Numero di sequenze diverse di lunghezza $k$\\
    $2^{k+1} - 2$ & Numero di sequenze di lunghezza massima $k$\\
    $log_2(N+2)-1$ & Lunghezza della sequenza piu' lunga rappesen
\end{tabular}
% TODO: NO BASTA, MI ARRENDO!, FANCULO.
