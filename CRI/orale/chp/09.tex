\chapter{La firma digitale}

Ai protocolli crittografici sono richieste tre funzionalit\'a importanti

\begin{itemize}
    \item Identificazione : Un sistema deve essere in grado di accertare l'identit\'a di un utente che richiede di accedere ai suoi servizi
    \item Autenticazione : Il destinatario di un messaggio deve essere in grado di accertare l'identit\'a del mittente e l'integrit\'a di un crittogramma ricevuto
    \item Firma digitale : Come una firma manuale, deve possedere tre caratteristiche (accertabili anche da una terza parte facente da giudice)
    \begin{itemize}
        \item Il mittente non ha facolt\'a di ripudiare il messaggio
        \item Il destinatario deve poter accertare l'identit\'a del mittente e l'integrit\'a del crittogramma
        \item Il destinatario non deve poter sostenere che $m' \not= m$ \'e il messaggio inviatogli
    \end{itemize}
\end{itemize}

\section{Funzioni hash one-way}

\begin{itemize}
    \item Una funzione hash $f : X \to Y$ \'e definita per $n = |X| \gg m = |Y|$
    \item La differenza di cardinalit\'a tra $X$ e $Y$ implica che esiste una partizione di $X$ in sottoinsiemi disgiunti $X_1,\dots,X_m \mid \forall i \in [1,m]$ tutti gli elementi in $X_i$ hanno come immagine uno stesso elemento in $Y$
    \item Una funzion hash one-way deve soddisfare tre caratteristiche
    \begin{itemize}
        \item Per ogni $x \in X$ \'e computazionalmente facile calcolare $f(x)$
        \item Per la maggior parte degli $y \in Y$ \'e computazionalmente difficile determinare $x$ tale che $f(x) = y$ (one way)
        \item \'E computazionalmente difficile determinare una coppia $x', x''$ in $X$ tale che $f(x') = f(x'')$ (claw free)
    \end{itemize}
\end{itemize}

\section{Identificazione}

\begin{itemize}
    \item Le password in un database vengono salvate sotto forma di hash con il seguente meccanismo
    \item Durante la fase di registrazione un utente sceglie una password $P_u$
    \item Il sistema genera un seme $S_u$ e calcola un hash $Q_u = f(P_uS_u)$ e memorizza $\abracket{u, S_u, Q_u}$
    \item Quando un utente prova ad effettuare l'accesso, il sistema recupera il suo seme e calcola $Q'_u$
    \item Se $Q'_u = Q_u$ l'identificazione ha avuto successo
    \item Alla password viene aggiunto un seme perch\`e altrimenti password uguali genererebbero hash uguali
\end{itemize}

\subsection{Identificazione tramite RSA}

\begin{itemize}
    \item Il sistema $S$ genera un valore casuale $r < n$ e lo invia in chiaro a $U$
    \item $U$ applica la sua chiave privata r calcolando $f = r^d \mod{n}$
    \item $S$ verifica l'identit\'a di $U$ applicando la chiave pubblica di $U$ stesso verificando che $f^e \mod{n} = r$
\end{itemize}

Questo protocollo per\`o richiede che $U$ si fidi di $S$. Infatti se $S$ non fosse \textit{chi dice di essere}, potrebbe richiedere a $U$ di applicare la propria chiave privata a messaggi opportunamente creati per inferire informazioni su $d$

\section{Autenticazione}

\begin{itemize}
    \item Il meccanismo di autenticazione pu\'o essere descritto attraverso una funzione $\mathcal{A}(m, k)$ che genera un'informazione (detta MAC) utile a garantire la provenienza e l'integrit\'a di m
    \item Se non \'e richiesta la confidenzialit\'a, \texttt{Mitt} spedisce la coppia\\$\abracket{m, \mathcal{A}(m, k)}$
    \item Altrimenti spedisce la coppia $\abracket{\mathcal{C}(m, k'), \mathcal{A}(m, k)}$
    \item Si pu\`o generare un MAC tramite una funzione hash one-way concatenando il messaggio e la chiave
    \item $\mathcal{A}(m, k) = h(mk)$
\end{itemize}

\section{Firma digitale}

\subsection{Messaggio cifrato e firmato in hash}

\begin{itemize}
    \item Il mittente $U$ calcola $f = \mathcal{D}(h(m), k_U[prv])$ e $c = \mathcal{C}(m, k_V[pub])$
    \item Spedisce la tripla $\abracket{U, c, f}$ a $V$
    \item $V$ calcola $m = \mathcal{D}(c, k_V)[prv]$ e $h(m) = \mathcal{C}(f, k_U[pub])$
    \item Se $h(m)$ \'e uguale al valore ottenibile ricalcolando la funzione hash sul messaggio, la firma \'e valida
\end{itemize}

\section{La Certification Authority}

\begin{itemize}
    \item Le \m{CA} autenticano le associazioni $\langle$ \m{utente, chiave pubblica} $\rangle$
    \item Un certificato contiene la chiave pubblica e una lista di informazioni relative al suo proprietario, tutto opportunamente firmato dalla \m{CA}
    \item Se \m{U} vuole comunicare con \m{V}, richiede cert$_V$ alla \m{CA}
\end{itemize}

\subsection{Messaggio cifrato, firmato in hash, e certificato}

\begin{itemize}
    \item \m{U} calcola $f$ e $c$ come nel protocollo senza certificato
    \item Spedisce la tripla $\abracket{cert_U, c, f}$
    \item \m{U} verifica $cert_U$ con la sua copia della chiave pubblica della \m{CA}
    \item Procede cone nel protocollo senza certificato
\end{itemize}

\section{Il protocollo SSL}

\section{Protocolli Zero-Knowledge}

\begin{itemize}
    \item Nei protocolli Zero-Knowledge due entit\'a (Prover $P$ e Verifier $V$) non si fidano l'uno dell'altro
    \item Il prover dovr\'a essere in grado di dimostrare al Verifier di essere in possesso di una facolt\'a particolare senza comunicargli alcuna informazione su di essa
    \item Questi protocolli si basano su una serie di iterazioni, dopo ognuna delle quali in Verifier aumenta la sua confidenza che $P$ sia \textit{chi dice di essere}
\end{itemize}