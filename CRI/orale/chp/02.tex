\chapter{Esercizi}

\section{Correttezza RSA}

\begin{itemize}
    \item $p, q \mid m$
    \begin{itemize}
        \item Eulero: $a^{\Phi(n)} \equiv 1 \mod{n}$
        \item $e \times d \equiv 1 \mod{\Phi(n)} = 1 + r\Phi(n)$
        \item $m^{ed} \mod{n} \equiv m \times m^{r\Phi(n)} \mod{n} \equiv m \times (m^{\Phi(n)})^r \mod{n} \\\dequiv{Eul} m \times 1^r \mod{n} \equiv m \mod{n}$
    \end{itemize}
    \item $p \mid m \wedge q \nmid m$
    \begin{itemize}
        \item $m \equiv m^r \equiv 0 \mod{p} \implies (m^r - m) \equiv 0 \mod{p}$
        \item Eulero: $a^{\Phi(q)} \equiv 1 \mod{q}$
        \item $m^{ed} \mod{q} \equiv m^{1+r\Phi{n}} \mod{q} \equiv m \times m^{r(p-1)(q-1)} \mod{q} \\\equiv m \times (m^{\Phi(q)})^{r\Phi(p)} \mod{q} \equiv m \mod{q}$
        \item $(m^{ed} - m) \equiv 0 \mod{q} \implies (m^{ed} - m) \mid q$
        \item $(m^{ed} - m) \equiv 0 \mod{n} \implies m^{ed} \equiv m \mod{n}$
    \end{itemize}
\end{itemize}

\section{Cifrario perfetto}

\subsection{Definizione}

\begin{itemize}
    \item Un cifrario si dice perfetto se non \'e possibile inferire alcuna informazione sul messaggio originale, dato il crittogramma associato
    \item $\forall m \in Msg, c \in Critto, \pr(M = m) = \pr(M = m \mid C = c)$
    \item La conoscenza complessiva di un crittoanalista non cambia dopo aver letto il crittogramma in transito sul canale
\end{itemize}

\subsection{Enunciato di Shannon}

\begin{itemize}
    \item Dati $M$, l'insieme dei messaggi possibili e $K$, l'insieme delle chiavi
    \item Per Shannon : $|K| \geq |M|$
    \item Poniamo per assurdo che $|K| < |M|$
    \item Fissato un crittogramma $c \mid \pr(C = c) > 0$, esso corrisponde a $s \leq |K|$ messaggi (non necessariamente distinti) in M
    \item Dato che $s \leq |K| < |M|$ allora necessariamente esiste almeno un messaggio $m \mid \pr(M=m)>0$ non ottenibile da $c$
    \item Quindi $\pr(M=m \mid C=c) = 0 \not= \pr(M=m)$
\end{itemize}

\subsection{Perfettezza di One-Time Pad}

\begin{itemize}
    \item $\pr(M=m \mid C=c) = \dfrac{\pr(M=m,C=c)}{\pr(C=c)}$
    \item Per definizione di \m{XOR}, fissato $m$, chiavi diverse corrispondono a crittogrammi diversi
    \item Perci\`o $\pr(C = c) = (\nicefrac{1}{2})^n$ \'e costante
    \item Quindi gli eventi $(C=c) e (M=m)$ sono indipendenti
    \item Ne risulta che $\pr(M=m \mid C=c) = \dfrac{\pr(M=m) \times \cancel{\pr(C=c)}}{\cancel{\pr(C=c)}}$
\end{itemize}

\section{RSA: n uguale per pi\'u utenti}

\begin{itemize}
    \item $\abracket{e_1, m} \abracket{e_2, m}$
    \item $mcd(e1, e2) = 1 \implies e_1r + e_2s = 1$
    \item $m = m^{e_1r + e_2s} = c_1^r \times c_2s \de{$r < 0$} (c_1^{-1})^{-r} \times c_2^s$
\end{itemize}

\section{RSA: e piccolo}

\begin{itemize}
    \item Ipotizziamo di avere $e$ utenti che ricevono lo stesso messaggio $m$
    \item $c_i = m^e \mod{n_i}$
    \item Per teorema cinese del resto
    \begin{itemize}
        \item $n = \prod_{i=1}^{e} n_i$
        \item $c' = m^e < n$
    \end{itemize}
    \item $\sqrt[e]{c'} = m$
\end{itemize}

\section{Zero knowledge}

\begin{itemize}
    \item $s < n \mid t = s^2 \mod{n}$
    \item $\abracket{s}: priv \mid \abracket{t, n}: pub$
    \item Iterazioni:
    \begin{itemize}
        \item $P : r < n, u = r^2 \mod{n} \rightarrow V$
        \item $V : e \in [0, 1] \rightarrow P$
        \item $P : z = rs^e \mod{n}$
        \item $V : z^2 \mod{n} \de{Verifica} ut^e \mod{n}$
    \end{itemize}
\end{itemize}

\section{Correttezza RSA}

\begin{itemize}
    \item $p, q \nmid m$
    \begin{itemize}
        \item $mcd(m, n) = 1$
        \item $a^\Pn \dequiv{Euler} 1 \mod{n}$
        \item $e \times d = 1 \mod{\Pn} = 1 + r\Pn$
        \item $m^{ed} \mod{n} \equiv m^{1 + r\Pn} \mod{n} \equiv m \times (m^\Pn)^r \mod{n} \dequiv{Euler} m \times 1^r \mod{n} = m$
    \end{itemize}
    \item $p \mid m \wedge q \nmid m$
    \begin{itemize}
        \item $m \equiv m^r \equiv 0 \mod{p}$
        \item $m^{ed} \mod{q} \equiv m \times m^{r\Pn} \mod{q} \equiv m \times (m^\Pk{q})^{r(p-1)} \mod{q} \dequiv{Euler} m \times 1^{r(p-1)} \mod{q} \equiv m \mod{q}$
        \item $m^{ed} \equiv m \mod{q} \implies (m^{ed} - m) \equiv 0 \mod{q}$
        \item $(m^{ed} - m) \equiv 0 \mod{n} \implies m^{ed} \equiv m \mod{n}$
    \end{itemize}
\end{itemize}

\section{Esistenza di numeri casuali di lunghezza n}

\begin{itemize}
    \item $S = $ \#\{Sequenze di lunghezza $\leq n$\} $= 2^n$
    \item $T = $ \#\{Sequenze non casuali in $S$\}
    \item $N = $ \#\{Sequenze di lunghezza $\leq n - \ceil{log_2n}$\} $= 2^{n - \ceil{log_2n}}$
    \item In $N$ ci sono anche gli algoritmi che generano i valori in $T$
    \item $T \leq N < S$
    \item $\frac{T}{S} < \frac{1}{2^{\ceil{\log_2n}}}$
\end{itemize}

\section{Definizione di cifrario perfetto}

\begin{itemize}
    \item Si considera la variabile aleatoria $M$ che assume valori nello spazio dei messaggi $Msg$
    \item Si considera la variabile aleatoria $C$ che assume valori nello spazio dei crittogrammi $Critto$
    \item $\Pr(M = m)$ \'e la probabilit\'a che $Mitt$ voglia spedire il messaggio $m$ a $Dest$
    \item $\Pr(M = m \mid C = c)$ \'e la probabilit\'a a posteriori che il messaggio fosse effettivamente $m$ dopo aver visto $c$ transitare sul canale
    \item Un cifrario si dice perfetto se $\Pr(M = m) = \Pr(M = m \mid C = c)$ cio\'e che la conoscenza complessiva di un crittoanalista non cambia dopo aver visto il crittogramma $c$
\end{itemize}

\section{Teorema di Shannon}

\begin{itemize}
    \item $N_m = \#\{m \mid \Pr(M = m) > 0\}$ 
    \item In un cifrario perfetto $N_k \geq N_m$
    \item Poniamo per assurdo che $N_k < N_m$
    \item Fissato $c \mid \Pr(C = c) > 0$, ci sono $s \leq N_k$ messaggi in (non necessariamente distinti) ottenuti decrittando $c$ con tutte le chavi
    \item Dato che $s \leq N_k < N_m$, esiste almeno un messaggio $m' \mid \Pr(M = m') > 0$ non ottenibile da $c$
    \item $\Pr(M = m \mid C = c) = 0 \not= \Pr(M = m)$
\end{itemize}

\section{Perfettezza One-Time Pad}

\begin{itemize}
    \item $\Pr(M = m \mid C = c) \de{Cond} \dfrac{\Pr(M = m, C = c)}{\Pr(C = c)}$
    \item Fissato un $m$, per definizione di \m{XOR} chiavi diverse generano crittogrammi diversi
    \item Perci\`o $\Pr(C = c) = (\frac{1}{2})^n$ costante
    \item \dots
\end{itemize}

\newpage
\section{Correttezza RSA}

\begin{itemize}
    \item $p, q \mid m$
    \begin{itemize}
        \item $mcd(m, n) = 1$
        \item $e \times d \equiv 1 \mod{\Pn} = 1 + r\Pn$
        \item $a^\Pn \dequiv{Euler} 1 \mod{n}$
        \item $m^{ed} \mod{n} = m^{1 + r\Pn} \mod{n} = m \times (m^\Pn)^r  \mod{n} \de{Euler} m \times 1^r \mod{n} = m \mod{n} = m$
    \end{itemize}
    \item $p \mid m \wedge q \nmid m$
    \begin{itemize}
        \item \dots
    \end{itemize}
\end{itemize}

\section{Fiat Shamir}

\begin{itemize}
    \item $P$ sceglie $s < n$ e calcola $t = s^2 \mod{n}$
    \item $\abracket{s}: Prv$
    \item $\abracket{t, n}: Pub$
    \begin{itemize}
        \item $P: r < n, u = r^2 \mod{n} \to V$
        \item $V: e \in [0, 1]$
        \item $P: z = us^e \mod{n} \to V$
        \item $V: z^2 \mod{n} \de{Ver} ut^e \mod{n}$
    \end{itemize}
\end{itemize}

\section{Algoritmo di Koblitz}

\begin{itemize}
    \item Si fissa $h \mid (m+1)h < p$
    \item Si itera $i : 1 \to h-1$
    \begin{itemize}
        \item $x = mh + i$
        \item Se $y^2 = x^3 + ax + b$ \'e definito sulla curva si prende $P_m = (x, y)$
        \item Altrimenti si itera
    \end{itemize}
\end{itemize}

\newpage
\section{ElGamal}

\begin{itemize}
    \item $E_p(a, b)$ curva
    \item $B$ punto sulla curva di ordine $n$ molto alto
    \item Crittazione
    \begin{itemize}
        \item $r < n$
        \item $V = rB, W = P_m + rP_D$
    \end{itemize}
    \item Decrittazione
    \begin{itemize}
        \item $P_m = W - n_DV = P_m + \cancel{rn_DB} - \cancel{n_DrB}$
    \end{itemize}
\end{itemize}